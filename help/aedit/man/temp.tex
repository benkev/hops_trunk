\begin{tabbing}
Titlexxxxxxxxxxxxxxx \= \kill
\underline{COMMAND} \> {\bf 	axis} \\
\end{tabbing}

\begin{tabbing}
Titlexxxxxxxxxxxxxxx \= \kill
\underline{TYPE} \> {\bf 		Plot control} \\
\end{tabbing}

\begin{tabbing}
Titlexxxxxxxxxxxxxxx \= \kill
\underline{SYNTAX} \> {\bf 		``axis snr"} \\
\> {\bf 		``axis amplitude"} \\
\> {\bf 		``axis phase"} \\
\> {\bf 		``axis sbdelay"} \\
\> {\bf 		``axis mbdelay <modulo>"} \\
\> {\bf 		``axis drate"} \\
\> {\bf 		``axis refdpcal"    (reference station phasecal difference)} \\
\> {\bf 		``axis remdpcal"    (remote ...)} \\
\> {\bf 		``axis refpcal1"    (reference station channel 1 phasecal)} \\
\> {\bf 		``axis rempcal1"    (remote ...)} \\
\end{tabbing}

\underline{DESCRIPTION}
\begin{list}{}{\setlength{\leftmargin}{0.5in}
     \setlength{\rightmargin}{0in}}
\item
Determines what will be plotted on the Y-axis.  The default
is ``amplitude".  The ``modulo" argument for multiband delay sets
the number of nanoseconds assumed between ambiguities.  The
plot is automatically scaled to +/- half this value.  If omitted,
``modulo" is assigned the value 500.
\item
\end{list}
\vspace{.2in}

\begin{tabbing}
Titlexxxxxxxxxxxxxxx \= \kill
\underline{COMMAND} \> {\bf 	baselines} \\
\end{tabbing}

\begin{tabbing}
Titlexxxxxxxxxxxxxxx \= \kill
\underline{TYPE} \> {\bf 		Data selection} \\
\end{tabbing}

\begin{tabbing}
Titlexxxxxxxxxxxxxxx \= \kill
\underline{SYNTAX} \> {\bf 		``baselines AB BC CD AC"} \\
\end{tabbing}

\underline{DESCRIPTION}
\begin{list}{}{\setlength{\leftmargin}{0.5in}
     \setlength{\rightmargin}{0in}}
\item
Sets the baseline data selection parameter in the inputs.  Only
those baselines specified will pass the filter-applying operations
of edit inputs, read, and plot.  Typing ``baselines" without
arguments removes any restrictions on allowed baselines.
\end{list}
\vspace{.2in}

\begin{tabbing}
Titlexxxxxxxxxxxxxxx \= \kill
\underline{COMMAND} \> {\bf 	batch} \\
\end{tabbing}

\begin{tabbing}
Titlexxxxxxxxxxxxxxx \= \kill
\underline{TYPE} \> {\bf 		Miscellaneous} \\
\end{tabbing}

\begin{tabbing}
Titlexxxxxxxxxxxxxxx \= \kill
\underline{SYNTAX} \> {\bf 		``batch"} \\
\end{tabbing}

\underline{DESCRIPTION}
\begin{list}{}{\setlength{\leftmargin}{0.5in}
     \setlength{\rightmargin}{0in}}
\item
Disables confirmation mechanism, for running in batch mode.
\end{list}
\vspace{.2in}

\begin{tabbing}
Titlexxxxxxxxxxxxxxx \= \kill
\underline{COMMAND} \> {\bf 	clear} \\
\end{tabbing}

\begin{tabbing}
Titlexxxxxxxxxxxxxxx \= \kill
\underline{TYPE} \> {\bf 		Action} \\
\end{tabbing}

\begin{tabbing}
Titlexxxxxxxxxxxxxxx \= \kill
\underline{SYNTAX} \> {\bf 		``clear data"} \\
\> {\bf 		``clear inputs"} \\
\> {\bf 		``clear plot"} \\
\> {\bf 		``clear all"} \\
\end{tabbing}

\underline{DESCRIPTION}
\begin{list}{}{\setlength{\leftmargin}{0.5in}
     \setlength{\rightmargin}{0in}}
\item
``Clear data" removes all the data from memory, and returns
array space to the system.  Since any active plot no longer
refers to data in memory after this operation, the plot is
rendered inactive.
\item
``Clear inputs" changes all the values listed by the ``inputs"
command to their default values.  Typically, this means
data selection parameters are set to pass all data, and plots
revert to self-scaling.
\item
``Clear plot" flushes the current plot, and renders a plot on
an interactive device inactive.
\item
``Clear all" simultaneously performs all the above operations.
\end{list}
\vspace{.2in}

\begin{tabbing}
Titlexxxxxxxxxxxxxxx \= \kill
\underline{COMMAND} \> {\bf 	device} \\
\end{tabbing}

\begin{tabbing}
Titlexxxxxxxxxxxxxxx \= \kill
\underline{TYPE} \> {\bf 		IO control} \\
\end{tabbing}

\begin{tabbing}
Titlexxxxxxxxxxxxxxx \= \kill
\underline{SYNTAX} \> {\bf 		``device name/device"} \\
\end{tabbing}

\underline{DESCRIPTION}
\begin{list}{}{\setlength{\leftmargin}{0.5in}
     \setlength{\rightmargin}{0in}}
\item
Sets the device type used for plotting.  The available devices
are those accessible to the PGPLOT library, which as of April
1990 consisted of tektronix emulators, a SunView graphics window,
and Hewlett-Packard Laserjet printers, 150 and 300 dots-per-inch.
\item
The construction of the argument is in two parts.  The first
part is the specific name of the output file or device,  The
second part specifies the type of device.  The former can
be a standard UNIX filename, such as ``plot01.3C345", but 
subdirectory specifiers (i.e. filenames with ``/" in them) are
special because PGPLOT is looking for a ``/" to separate
the two parts of the device specifier.  You must ``hide" the
UNIX ``/" characters from PGPLOT by enclosing the filename in
double quotes, so that a valid specification for a workstation
tektronix emulator might be ```/dev/ttyp2"/te'.
\item
The default filename for interactive devices is the users
terminal, whilst for the hardcopy devices, it is ``PGPLOT.device".
PGPLOT translates filenames to upper case on output.
\item
The second part, the device type, follows a ``/", and presently
may be ``null" (bit bucket), ``tek4010", ``sun" (on Sun workstations), 
``ljlowres", or ``ljhighres". These names are minimum matchable 
(e.g. ``/te" will work).  
\item
The initial state of the ``device" input parameter is ``?", which
switches on an interactive device-setting mode inside PGPLOT.  The
response to the PGPLOT query is then automatically inserted into
the aedit device parameter.
\end{list}
\vspace{.2in}

\begin{tabbing}
Titlexxxxxxxxxxxxxxx \= \kill
\underline{COMMAND} \> {\bf 	edit} \\
\end{tabbing}

\begin{tabbing}
Titlexxxxxxxxxxxxxxx \= \kill
\underline{TYPE} \> {\bf 		Action} \\
\end{tabbing}

\begin{tabbing}
Titlexxxxxxxxxxxxxxx \= \kill
\underline{SYNTAX} \> {\bf 		``edit inputs"} \\
\> {\bf 		``edit cursor"} \\
\> {\bf 		``edit duplicates procdate"} \\
\> {\bf 		``edit duplicates qcode"} \\
\> {\bf 		``edit duplicates snr"} \\
\end{tabbing}

\underline{DESCRIPTION}
\begin{list}{}{\setlength{\leftmargin}{0.5in}
     \setlength{\rightmargin}{0in}}
\item
Sets flags in the data records according to a variety of
circumstances.  These flags can be selectively unset with
the ``unflag" command.
\item
``Edit inputs" sets a flag bit in each data record for each
data selection input parameter which excludes that data point.
Thus, a scan may pass the input filter for stations, but fail
that for baselines.  The baseline bit would be set, but the
station bit would not.  Any set bit in the flag field causes
the scan to be flagged (i.e. it will not be plotted or written
to an output file).
\item
``Edit cursor" enables the cursor on an interactive graphics device
upon which data has been displayed using ``plot".  The user may
type any character (except `x', `X', `a', `A', `b' or `B' ..  see 
below) on the keyboard to edit out the point nearest the cursor.
The cursor must be inside the border of a plot, and must be twice 
as close to the target point than any other point for success.  
Failure to meet these conditions results in an appropriate error 
message.    
\item
Alternatively, the user may define an area on the plot within which
all points are to be edited out.  This is accomplished by typing
`a' or `A' to locate the bottom left corner of a rectangle, and
`b' or `B' to locate the top right corner. 
\item
On devices which are not capable of erasing points from the screen
(e.g. tektronix emulators), the edited points are marked by being
overwritten by a solid square.
\item
Do not use the mouse buttons on workstation tektronix emulators - 
these return multiple characters which may confuse the program.
The cursor editing mode is terminated by typing the character `x' 
or `X' on the keyboard.
\item
``Edit duplicates" removes duplicate scans from the database, 
ignoring flagged scans.  The term ``duplicate" refers to identical
baseline, scan time, frequency code, experiment number and source.
The second argument determines which scan aedit will retain.  If
``procdate" is specified, it will keep the most recent processing.
If ``qcode" is specified, the ``best" quality code scan is kept. If
``snr" is specified, the highest snr scan is kept.
\item
WARNING: Since ``edit duplicates" ignores flagged scans, unflagging
data may generate more duplicates.  Similarly, reading in more
data may do the same.  In such circumstances, the recommended
course is to ``unflag duplicates" and rerun ``edit duplicates".
\end{list}
\vspace{.2in}

\begin{tabbing}
Titlexxxxxxxxxxxxxxx \= \kill
\underline{COMMAND} \> {\bf 	exit} \\
\end{tabbing}

\begin{tabbing}
Titlexxxxxxxxxxxxxxx \= \kill
\underline{TYPE} \> {\bf 		Action} \\
\end{tabbing}

\begin{tabbing}
Titlexxxxxxxxxxxxxxx \= \kill
\underline{SYNTAX} \> {\bf 		``exit" (no arguments)} \\
\end{tabbing}

\underline{DESCRIPTION}
\begin{list}{}{\setlength{\leftmargin}{0.5in}
     \setlength{\rightmargin}{0in}}
\item
Terminates the current aedit session.  All data currently in memory
is lost.  The plot device, if open, is closed and the plot flushed.
\end{list}
\vspace{.2in}

\begin{tabbing}
Titlexxxxxxxxxxxxxxx \= \kill
\underline{COMMAND} \> {\bf 	experiment} \\
\end{tabbing}

\begin{tabbing}
Titlexxxxxxxxxxxxxxx \= \kill
\underline{TYPE} \> {\bf 		Data selection} \\
\end{tabbing}

\begin{tabbing}
Titlexxxxxxxxxxxxxxx \= \kill
\underline{SYNTAX} \> {\bf 		``experiment 1953"} \\
\end{tabbing}

\underline{DESCRIPTION}
\begin{list}{}{\setlength{\leftmargin}{0.5in}
     \setlength{\rightmargin}{0in}}
\item
Sets the experiment input data selection parameter.  Only one
experiment number may be specified at one time.  Scans which do
not belong to the specified experiment number will not pass the
filters applied by edit inputs, read, and plot.  Typing ``experiment"
without arguments removes any restriction on experiment number.
\end{list}
\vspace{.2in}

\begin{tabbing}
Titlexxxxxxxxxxxxxxx \= \kill
\underline{COMMAND} \> {\bf 	fraction} \\
\end{tabbing}

\begin{tabbing}
Titlexxxxxxxxxxxxxxx \= \kill
\underline{TYPE} \> {\bf 		Data selection} \\
\end{tabbing}

\begin{tabbing}
Titlexxxxxxxxxxxxxxx \= \kill
\underline{SYNTAX} \> {\bf 	  e.g.	``fraction > 8"} \\
\> {\bf 	    or	``fraction <= 60%"} \\
\end{tabbing}

\underline{DESCRIPTION}
\begin{list}{}{\setlength{\leftmargin}{0.5in}
     \setlength{\rightmargin}{0in}}
\item
Sets the fraction of the data processed for this scan which
will pass the filtering functions applied in read, edit, and
plot.  The syntax is quite forgiving.  The requirements are
that there be an inequality operator, possibly followed by
an equals sign, followed by a sensible number, possibly followed
by a percent sign.  If the percent sign is missing, the number
is interpreted as tenths of the scheduled data, instead of a 
percentage.  Spaces are irrelevant.
\item
If ``fraction" is typed with no arguments, or with
just ``0" as an argument, all restrictions on the fraction of
data processed are removed.
\item
Note that this filter option operates on the value of the last
digit in the ESDESP field of the A-file format, which is placed
there by FRNGE.  Before the implementation of baseline-dependent
scan lengths in the schedule files, this number was unreliable.
Also, being only a single digit, this quantity is only accurate
to the nearest 10%, so more precise values entered with the 
fraction command are rounded off.
\item
\end{list}
\vspace{.2in}

\begin{tabbing}
Titlexxxxxxxxxxxxxxx \= \kill
\underline{COMMAND} \> {\bf 	frequencies} \\
\end{tabbing}

\begin{tabbing}
Titlexxxxxxxxxxxxxxx \= \kill
\underline{TYPE} \> {\bf 		Data selection} \\
\end{tabbing}

\begin{tabbing}
Titlexxxxxxxxxxxxxxx \= \kill
\underline{SYNTAX} \> {\bf 		``frequencies XS, C"} \\
\end{tabbing}

\underline{DESCRIPTION}
\begin{list}{}{\setlength{\leftmargin}{0.5in}
     \setlength{\rightmargin}{0in}}
\item
Enters a list of allowed frequencies into the inputs.  All alphabetic
characters are accepted, in any order, lower or upper case, with or
without spaces or commas.  Duplicate characters are ignored.  Scans
which involve frequencies not in this list will fail the filter tests
applied by edit inputs, read, and plot.  Typing ``frequencies" without
arguments removes any limitations on frequencies.
\end{list}
\vspace{.2in}

\begin{tabbing}
Titlexxxxxxxxxxxxxxx \= \kill
\underline{COMMAND} \> {\bf 	grid} \\
\end{tabbing}

\begin{tabbing}
Titlexxxxxxxxxxxxxxx \= \kill
\underline{TYPE} \> {\bf 		Plot control} \\
\end{tabbing}

\begin{tabbing}
Titlexxxxxxxxxxxxxxx \= \kill
\underline{SYNTAX} \> {\bf 		``grid n1 n2" (n1, n2 are integers - n1 <= 3, n2 <= 10)} \\
\end{tabbing}

\underline{DESCRIPTION}
\begin{list}{}{\setlength{\leftmargin}{0.5in}
     \setlength{\rightmargin}{0in}}
\item
This sets the parameter which determines how many subplots
appear horizontally and vertically on the plotting surface.
The default is one in each direction, the maximum is 3
horizontally and 10 vertically.  The character size scales
with the number of vertical plots to keep things readable.
\end{list}
\vspace{.2in}

\begin{tabbing}
Titlexxxxxxxxxxxxxxx \= \kill
\underline{COMMAND} \> {\bf 	help} \\
\end{tabbing}

\begin{tabbing}
Titlexxxxxxxxxxxxxxx \= \kill
\underline{TYPE} \> {\bf 		Action} \\
\end{tabbing}

\begin{tabbing}
Titlexxxxxxxxxxxxxxx \= \kill
\underline{SYNTAX} \> {\bf 		``help `subject'"} \\
\end{tabbing}

\underline{DESCRIPTION}
\begin{list}{}{\setlength{\leftmargin}{0.5in}
     \setlength{\rightmargin}{0in}}
\item
Writes the help file pertaining to ``subject" on the 
terminal, under pagination control.  ``Subject" is
presently any command name, plus ``general".
\item
Action commands:
\item
	Name	argument(s)	Description
	----	-----------     -----------
	clear	data		Erase all data from memory
		inputs		Reset input settings to default
		plot		Clear screen or eject page
		all		All three clear functions
	edit	cursor		Zap points on screen with cursor
		inputs		Remove points that don`t fit inputs
		duplicates	Remove duplicate points with various priorities
	exit			End aedit session
	inputs	plot/filter	Print current input settings on screen
	plot			Plot current data according to inputs
	read	filename	Read in data from filename
	run	filename	Execute commands in filename
	setyear	number		Manually reset year of scan throughout data
	sort	key		Sort data according to various keys
	summary			Display a summary of all unflagged data
	unflag	string		Removes flags applied for various reasons
	unsort			Restore original sort order (as read in)
	write	filename	Write (edited, sorted) data to filename
	zoom			Display details of cursor-selected points
\item
Plot control parameters:
\item
	Name	argument(s)	Description
	----	-----------     -----------
	axis	string		Set variable to plot on Y axis
	grid	a,b		Divide screen/page a times b subplots
	scale	min,max		Set Y-axis scale
\item
Data selection parameters:
\item
	Name		argument(s)	Description
	----		-----------     -----------
	baselines	AB,BC,AC ....	Use only these baselines
	experiment	expt #		Use only data from this experiment
	frequencies	S,X,K, ....	Use only data at these frequencies
	fraction	nn% ....	Use only scans with >nn% good data
	length		number		Use only scans > number secs or blocks
	nfreq		<>= nn		Use only scans with <>= nn frequencies
	qcodes		5-9,D ...	Use only data with these quality codes
	snrmin		number		Use only data with snr > number
	sources		name1,name2...  Use only data on these sources
	stations	A,B,C,D ....	Use only data from these stations
	timerange	d-hhmm, d-hhmm	Use only data in time range
	type		0 1 2		Use only data of these types
\item
I/O control parameters:
\item
	Name	argument(s)	Description
	----	-----------     -----------
	device	string		Plotting device for PGPLOT
\item
Miscellaneous:
\item
	Name	argument(s)	Description
	----	-----------	-----------
	batch	none		Disables interactive confirmation querys
	nobatch	none		Enables interactive confirmation querys
\end{list}
\vspace{.2in}

\begin{tabbing}
Titlexxxxxxxxxxxxxxx \= \kill
\underline{COMMAND} \> {\bf 	inputs} \\
\end{tabbing}

\begin{tabbing}
Titlexxxxxxxxxxxxxxx \= \kill
\underline{TYPE} \> {\bf 		Action} \\
\end{tabbing}

\begin{tabbing}
Titlexxxxxxxxxxxxxxx \= \kill
\underline{SYNTAX} \> {\bf 		``inputs plot"} \\
\> {\bf 	    or  ``inputs filter"} \\
\> {\bf 	or just ``inputs"} \\
\end{tabbing}

\underline{DESCRIPTION}
\begin{list}{}{\setlength{\leftmargin}{0.5in}
     \setlength{\rightmargin}{0in}}
\item
For use in ascii-terminal interface mode only.  Places a
summary of the current aedit input settings on the screen.
The plot and filter options result in a display of only 
those inputs pertaining to plotting and data filtering
respectively, while the default produces a display of all
input parameters.
\item
An example is shown below.
\item
	****************
	| AEDIT INPUTS |
	****************
\item
DATA SELECTION PARAMETERS
-------------------------
Timerange:   88124-1257 to  88126-0715
Stations:    ABNT
Baselines:   AB BN TN 
Frequencies: XS
Experiment:  1996
Qcodes:      56789AD
Type:        2
Snrmin:      10
Sources:     3C345 3C273 OJ287
Length:      30 
Fraction:    No restriction specified
\item
PLOTTING PARAMETERS
-------------------
Axis:        Plot mbdelay against time (Modulo 40 nanoseconds)
Grid:        divide plot into 2 horizontal and 5 vertical subplots
Scale:       Plot between mbdelay = -20.000000 and 20.000000
Mode:        Nosplit (multiple sources per plot)
Device:      Device for graphics output = ``/dev/ttyp2"/te
\item
\end{list}
\vspace{.2in}

\begin{tabbing}
Titlexxxxxxxxxxxxxxx \= \kill
\underline{COMMAND} \> {\bf 	length} \\
\end{tabbing}

\begin{tabbing}
Titlexxxxxxxxxxxxxxx \= \kill
\underline{TYPE} \> {\bf 		Data selection} \\
\end{tabbing}

\begin{tabbing}
Titlexxxxxxxxxxxxxxx \= \kill
\underline{SYNTAX} \> {\bf 		``length 20"} \\
\end{tabbing}

\underline{DESCRIPTION}
\begin{list}{}{\setlength{\leftmargin}{0.5in}
     \setlength{\rightmargin}{0in}}
\item
Sets the minimum scan length in seconds (type 2 data),
or HP-1000 256-byte blocks (type 0 and 1 data), 
which will pass the filters applied in edit inputs, 
read, and plot.  Typing ``length" without arguments 
removes any limitation on scan length.
\end{list}
\vspace{.2in}

\begin{tabbing}
Titlexxxxxxxxxxxxxxx \= \kill
\underline{COMMAND} \> {\bf 	mode} \\
\end{tabbing}

\begin{tabbing}
Titlexxxxxxxxxxxxxxx \= \kill
\underline{TYPE} \> {\bf 		plot control} \\
\end{tabbing}

\begin{tabbing}
Titlexxxxxxxxxxxxxxx \= \kill
\underline{SYNTAX} \> {\bf 		``mode split"} \\
\> {\bf 	    or  ``mode nosplit"} \\
\end{tabbing}

\underline{DESCRIPTION}
\begin{list}{}{\setlength{\leftmargin}{0.5in}
     \setlength{\rightmargin}{0in}}
\item
Toggles the setting of the mode parameter, which determines
whether or not the data will be split into one plot per
source.  The default on startup is ``nosplit".
\end{list}
\vspace{.2in}

\begin{tabbing}
Titlexxxxxxxxxxxxxxx \= \kill
\underline{COMMAND} \> {\bf 	nfreq} \\
\end{tabbing}

\begin{tabbing}
Titlexxxxxxxxxxxxxxx \= \kill
\underline{TYPE} \> {\bf 		Data selection} \\
\end{tabbing}

\begin{tabbing}
Titlexxxxxxxxxxxxxxx \= \kill
\underline{SYNTAX} \> {\bf 	  e.g.	``nfreq >= 8"} \\
\> {\bf 	    or	``nfreq < 2"} \\
\end{tabbing}

\underline{DESCRIPTION}
\begin{list}{}{\setlength{\leftmargin}{0.5in}
     \setlength{\rightmargin}{0in}}
\item
Sets the number of frequencies processed for this scan which
will pass the filtering functions applied in read, edit, and
plot.  The syntax is quite forgiving.  The requirements are
that there be an optional inequality operator, possibly followed 
by an equals sign, followed by a sensible number.  Spaces are 
irrelevant.  If the inequality is omitted, exactly the specified
number of frequencies must be present to pass the filters.
\item
If ``nfreq" is typed with no arguments, or with just ``0" as an 
argument, all restrictions on the number of frequencies 
processed are removed.
\item
\end{list}
\vspace{.2in}

\begin{tabbing}
Titlexxxxxxxxxxxxxxx \= \kill
\underline{COMMAND} \> {\bf 	nobatch} \\
\end{tabbing}

\begin{tabbing}
Titlexxxxxxxxxxxxxxx \= \kill
\underline{TYPE} \> {\bf 		Miscellaneous} \\
\end{tabbing}

\begin{tabbing}
Titlexxxxxxxxxxxxxxx \= \kill
\underline{SYNTAX} \> {\bf 		``nobatch"} \\
\end{tabbing}

\underline{DESCRIPTION}
\begin{list}{}{\setlength{\leftmargin}{0.5in}
     \setlength{\rightmargin}{0in}}
\item
Enables confirmation mechanism, for running interactively
(reverses the action of ``batch").
\end{list}
\vspace{.2in}

\begin{tabbing}
Titlexxxxxxxxxxxxxxx \= \kill
\underline{COMMAND} \> {\bf 	parameter} \\
\end{tabbing}

\begin{tabbing}
Titlexxxxxxxxxxxxxxx \= \kill
\underline{TYPE} \> {\bf 		Action} \\
\end{tabbing}

\begin{tabbing}
Titlexxxxxxxxxxxxxxx \= \kill
\underline{SYNTAX} \> {\bf 		``parameter 1 2 3 ..."	(non-interactive form)} \\
\> {\bf 		``parameter"		(interactive form)} \\
\end{tabbing}

\underline{DESCRIPTION}
\begin{list}{}{\setlength{\leftmargin}{0.5in}
     \setlength{\rightmargin}{0in}}
\item
This command causes all unedited type 2 data in memory to be treated as the
basis for a parameter extraction operation from disk-resident type-2 files.
Specified parameters are placed in a special array attached to each type-2 line
in memory.  The parameter specification is via key numbers.  These numbers
may be specified either directly on the input line of the parameter command,
or in response to a query from the program if no parameter keys are given.
In batch mode, aedit assumes that the former mechanism is being used, and
the absence of any keys is treated as an error.  Once extracted, the parameters
may be written to a file of the user`s choice, using the pwrite command.
\item
Obviously, aedit cannot extract parameters unless the relevant type-2 files are
on the disk.  Make sure the DATADIR environment variable is pointing to the 
correct data area.
\item
Each invocation of the parameter command obliterates all previous parameters
extracted for a previous subset of unflagged data lines.
\item
Below is a list of the available parameters, and their index numbers which
must be supplied in a space-delimited list.  The total number of parameters 
allowed is currently 32, and each array of parameters (denoted by the parentheses 
below) counts one for each array element
\item
INDEX    PARAMETER NAME
-----    --------------
 1:     ref_pcal_amp (nfreq)
 2:     ref_pcal_phase (nfreq)
 3:     ref_pcal_freq (nfreq)
 4:     ref_pcal_rate
 5:     rem_pcal_amp (nfreq)
 6:     rem_pcal_phase (nfreq)
 7:     rem_pcal_freq (nfreq)
 8:     rem_pcal_rate
 9:     errate_ref_usb (nfreq)
10:     errate_ref_lsb (nfreq)
11:     errate_rem_usb (nfreq)
12:     errate_rem_lsb (nfreq)
13:     corel_amp (nfreq)
14:     corel_phase (nfreq)
16:     rate_error
17:     mbdelay_error
18:     sbdelay_error
19:     total_phase
20:     tot_phase_mid
21:     incoherent_amp
22:     ref_elevation
23:     rem_elevation
24:     fr_arcsec_ns
25:     fr_arcsec_ew
26:     mhz_arcsec_ns
27:     mhz_arcsec_ew
28:     pcnt_discard
29:     min_max_ratio
30:     ref_frequency
31:     lo_frequency (nfreq)
\end{list}
\vspace{.2in}

\begin{tabbing}
Titlexxxxxxxxxxxxxxx \= \kill
\underline{COMMAND} \> {\bf 	plist} \\
\end{tabbing}

\begin{tabbing}
Titlexxxxxxxxxxxxxxx \= \kill
\underline{TYPE} \> {\bf 		action} \\
\end{tabbing}

\begin{tabbing}
Titlexxxxxxxxxxxxxxx \= \kill
\underline{SYNTAX} \> {\bf 		``plist"} \\
\end{tabbing}

\underline{DESCRIPTION}
\begin{list}{}{\setlength{\leftmargin}{0.5in}
     \setlength{\rightmargin}{0in}}
\item
This command summarizes the state of extracted parameters in memory.  Various
states can exist, with varying degrees of overlap between flagged and unflagged
records, with and without attached extracted parameters.  This command is
provided to remind the user of the degree to which he/she has confused
him/herself.
\item
A more important function is to attach a numerical identifying tag to each
extracted parameter present.  This tag is then used to identify the parameter
to be examined in filtering operations, using the ``prange" command.  You can
see whether you got the filter specification correct by using the ``inputs"
command after attempting to use ``prange".
\item
The id tag is the first field in the output of ``plist".
\end{list}
\vspace{.2in}

\begin{tabbing}
Titlexxxxxxxxxxxxxxx \= \kill
\underline{COMMAND} \> {\bf 	plot} \\
\end{tabbing}

\begin{tabbing}
Titlexxxxxxxxxxxxxxx \= \kill
\underline{TYPE} \> {\bf 		Action} \\
\end{tabbing}

\begin{tabbing}
Titlexxxxxxxxxxxxxxx \= \kill
\underline{SYNTAX} \> {\bf 		``plot" (no arguments)} \\
\end{tabbing}

\underline{DESCRIPTION}
\begin{list}{}{\setlength{\leftmargin}{0.5in}
     \setlength{\rightmargin}{0in}}
\item
Initiates plotting of data in memory on a device of the
users choice.  The data is plotted one baseline and one
frequency at a time, and is filtered by input settings.
If the ``mode" input parameter is set to ``split", the data
is further divided into one source per plot.
\item
The x-axis is always time, but the y-axis type is specified
by the ``axis" command to be one of several options.  The
y-axis extrema of the plot can be set with the ``scale"
command, and the layout of one or more plots on a page is
controlled with the ``grid" command.  You must call ``plot"
before using the cursor (``zoom" or ``edit cursor").
\end{list}
\vspace{.2in}

\begin{tabbing}
Titlexxxxxxxxxxxxxxx \= \kill
\underline{COMMAND} \> {\bf 	prange} \\
\end{tabbing}

\begin{tabbing}
Titlexxxxxxxxxxxxxxx \= \kill
\underline{TYPE} \> {\bf 		Data selection} \\
\end{tabbing}

\begin{tabbing}
Titlexxxxxxxxxxxxxxx \= \kill
\underline{SYNTAX} \> {\bf 		e.g. ``prange 2 >4.6"} \\
\> {\bf 		or ``prange 17 <1.5e-12"} \\
\> {\bf 		or ``prange 1 -25 74"} \\
\end{tabbing}

\underline{DESCRIPTION}
\begin{list}{}{\setlength{\leftmargin}{0.5in}
     \setlength{\rightmargin}{0in}}
\item
This command sets the input filter for a selected extracted parameter.  The
parameter is identified by the first argument, which is the identification tag
of the parameter reported by the ``plist" command.  It is thus not easy to run
``prange" without first executing ``plist".  Neither of these commands work,
obviously, unless you have already extracted some parameters with the
``parameter" command.
\item
The parameter data range of this filter can be specified either with an inequality
(no >= or <= because all parameters are floating point quantities internally),
or two numerical values, a lower and then an upper limit.  To exclude a finite
range of values, you must merge two input files, each of which has had one of
the inequality limits applied.  This limitation will be removed in due course.
Thus, the above examples will pass, respectively:
\item
All scans with values of extracted parameter 2 greater than 4.6
All scans with values of extracted parameter 17 less than 1.5e-12
All scans with values of extracted parameter 1 between -25.0 and +74.0
\item
In the same manner as all filter settings in aedit, the data flagging occurs only
upon invocation of the ``edit inputs" command.  In the case of extracted
parameters, the filtering during a read operation which normally occurs is
suppressed.
\end{list}
\vspace{.2in}

\begin{tabbing}
Titlexxxxxxxxxxxxxxx \= \kill
\underline{COMMAND} \> {\bf 	pwrite} \\
\end{tabbing}

\begin{tabbing}
Titlexxxxxxxxxxxxxxx \= \kill
\underline{TYPE} \> {\bf 		Action} \\
\end{tabbing}

\begin{tabbing}
Titlexxxxxxxxxxxxxxx \= \kill
\underline{SYNTAX} \> {\bf 		``pwrite filename"} \\
\end{tabbing}

\underline{DESCRIPTION}
\begin{list}{}{\setlength{\leftmargin}{0.5in}
     \setlength{\rightmargin}{0in}}
\item
Writes all unflagged user-extracted parameter data in memory out 
to the filename specified in the argument.  The data are written out
according to the current sort order (as determined by
execution of the ``sort" command).  If the data is not
sorted, the output order is the same as the order in
which the data was read.  You must execute the parameter command
before using pwrite.  Unflagged data lines which for any reason
do not have associated extracted parameters are ignored by pwrite.
\item
The list of user-extracted parameters is preceded by information
identifying the baseline, scan and extent number, together with
a few other generally useful items (but far less than is present
in the A-file format).
\end{list}
\vspace{.2in}

\begin{tabbing}
Titlexxxxxxxxxxxxxxx \= \kill
\underline{COMMAND} \> {\bf 	qcodes} \\
\end{tabbing}

\begin{tabbing}
Titlexxxxxxxxxxxxxxx \= \kill
\underline{TYPE} \> {\bf 		Data selection} \\
\end{tabbing}

\begin{tabbing}
Titlexxxxxxxxxxxxxxx \= \kill
\underline{SYNTAX} \> {\bf 		``qcodes 5,6,789,DEF"} \\
\> {\bf 	    or  ``qcodes 5-9 D-F"} \\
\> {\bf 	    or  ``qcodes not 0-4 A-C"} \\
\end{tabbing}

\underline{DESCRIPTION}
\begin{list}{}{\setlength{\leftmargin}{0.5in}
     \setlength{\rightmargin}{0in}}
\item
Sets the quality code data selection input parameter.  Shown
in the example are three ways of establishing the quality code
filter ``56789DEF".  You can specify codes directly, in any order,
separated by spaces, commas, or nothing at all.  You can also
specify ranges of quality codes from the sequence ``ABCDEF0123456789"
by using the construction ``2-8".  Preceding a specification by the
exact string ``not" means take all except the specified codes.  This
information is applied as a filter by edit inputs, read, and plot.
Typing ``qcodes" with no argument removes any limitation on quality
codes.
\end{list}
\vspace{.2in}

\begin{tabbing}
Titlexxxxxxxxxxxxxxx \= \kill
\underline{COMMAND} \> {\bf 	read} \\
\end{tabbing}

\begin{tabbing}
Titlexxxxxxxxxxxxxxx \= \kill
\underline{TYPE} \> {\bf 		Action} \\
\end{tabbing}

\begin{tabbing}
Titlexxxxxxxxxxxxxxx \= \kill
\underline{SYNTAX} \> {\bf 		``read filename"} \\
\end{tabbing}

\underline{DESCRIPTION}
\begin{list}{}{\setlength{\leftmargin}{0.5in}
     \setlength{\rightmargin}{0in}}
\item
Reads data in from the filename specified in the argument.
At present, only type 2 lines are decoded and stored in 
memory.  If all fields on the line are successfully decoded, 
the data enters memory with a zero flag field.  If some data 
could not be decoded, but enough is available to identify 
the parent data file, a flagged data entry is made in memory.  
If not even enough could be decoded to id the file, the line 
is skipped.  You can read as many files into aedit, one after 
the other, as you like.  ``Read" filters the incoming data 
according to the data selection input parameters.
\item
The unlimited data capacity of aedit is achieved by using
dynamic memory allocation inside the ``read" function.  As
more memory is needed, the program obtains it from the system.
This memory is released by the command ``clear data", or
by ``exit".  The user is informed of memory usage during the
reading operation.
\item
WARNING: The plot package PGPLOT also uses dynamic memory
allocation, which is inconsistent with subsequent memory
allocation by ``read".  If ``plot" has been executed, ``read"
will refuse to read in more data until a ``clear data" has
been issued.  This limitation may be removed in future
releases.
\end{list}
\vspace{.2in}

\begin{tabbing}
Titlexxxxxxxxxxxxxxx \= \kill
\underline{COMMAND} \> {\bf 	run} \\
\end{tabbing}

\begin{tabbing}
Titlexxxxxxxxxxxxxxx \= \kill
\underline{TYPE} \> {\bf 		action} \\
\end{tabbing}

\begin{tabbing}
Titlexxxxxxxxxxxxxxx \= \kill
\underline{SYNTAX} \> {\bf 		``run filename"} \\
\end{tabbing}

\underline{DESCRIPTION}
\begin{list}{}{\setlength{\leftmargin}{0.5in}
     \setlength{\rightmargin}{0in}}
\item
Causes the aedit commands in ``filename" to be executed, just
as if they were typed at the terminal.  For obvious reasons,
there are a couple of exceptions.  Confirmation is no longer
requested when using run files, and certain interactive
operations (edit cursor, zoom) are disabled.  Aedit command
files can be nested up to a depth of 10.  Any error within
a run file causes the execution to abort, and control returns
to the terminal, regardless of how deeply the runfiles are
nested.
\end{list}
\vspace{.2in}

\begin{tabbing}
Titlexxxxxxxxxxxxxxx \= \kill
\underline{COMMAND} \> {\bf 	scale} \\
\end{tabbing}

\begin{tabbing}
Titlexxxxxxxxxxxxxxx \= \kill
\underline{TYPE} \> {\bf 		Plot control} \\
\end{tabbing}

\begin{tabbing}
Titlexxxxxxxxxxxxxxx \= \kill
\underline{SYNTAX} \> {\bf 		``scale ymin ymax" (ymin, ymax floating point)} \\
\end{tabbing}

\underline{DESCRIPTION}
\begin{list}{}{\setlength{\leftmargin}{0.5in}
     \setlength{\rightmargin}{0in}}
\item
Sets the minimum and maximum Y-axis values, overriding the
default, which is 0 to 360 degrees for phase quantities, or
the data range plus 10% at each end for other quantities.
Scale with no arguments restores the default.  When
the Y-axis is multiband delay, the scale is automatically set
to +/- half the multiband delay ambiguity (user specified).
This can be overridden by an explicit ``scale" command.
\item
Points which fall outside the scale limits are not plotted, and
a warning message is issued to alert the user as to how many
points were omitted.
\end{list}
\vspace{.2in}

\begin{tabbing}
Titlexxxxxxxxxxxxxxx \= \kill
\underline{COMMAND} \> {\bf 	setyear} \\
\end{tabbing}

\begin{tabbing}
Titlexxxxxxxxxxxxxxx \= \kill
\underline{TYPE} \> {\bf 		Action} \\
\end{tabbing}

\begin{tabbing}
Titlexxxxxxxxxxxxxxx \= \kill
\underline{SYNTAX} \> {\bf 		``setyear 1989"} \\
\end{tabbing}

\underline{DESCRIPTION}
\begin{list}{}{\setlength{\leftmargin}{0.5in}
     \setlength{\rightmargin}{0in}}
\item
This command is present only to allow the user to circumvent an
unfortunate problem with the A-file format.  Some A-files have
the year of the scan in field 7, but in others this information
is replaced by the number of the parent type-51 HP-1000 extent.
Generally, aedit will recognize the latter type of A-file on
read, and notify the user that the scan year information is
missing from some of the data.  In such cases, the year is set
to 1980.  The recommended course of action is for the user to
set the timerange to the offending span in 1980 with all other
filters wide open, run ``edit inputs" to flag all good data,
force the year to the correct value with ``setyear", and unflag
the good data again.  If all data is actually from the same
calendar year, the edit and unflag steps are unnecessary - you
can run setyear on the whole dataset.
\item
Confusing things could happen if the parent extent number exceeds
80, but this should be almost never.
\end{list}
\vspace{.2in}

\begin{tabbing}
Titlexxxxxxxxxxxxxxx \= \kill
\underline{COMMAND} \> {\bf 	snrmin} \\
\end{tabbing}

\begin{tabbing}
Titlexxxxxxxxxxxxxxx \= \kill
\underline{TYPE} \> {\bf 		Data selection} \\
\end{tabbing}

\begin{tabbing}
Titlexxxxxxxxxxxxxxx \= \kill
\underline{SYNTAX} \> {\bf 		``snrmin 20"} \\
\end{tabbing}

\underline{DESCRIPTION}
\begin{list}{}{\setlength{\leftmargin}{0.5in}
     \setlength{\rightmargin}{0in}}
\item
Sets the minimum snr which will pass the filters applied
in edit inputs, read, and plot.  Typing ``snrmin" without
arguments removes any limitation on snr.
\end{list}
\vspace{.2in}

\begin{tabbing}
Titlexxxxxxxxxxxxxxx \= \kill
\underline{COMMAND} \> {\bf 	sort} \\
\end{tabbing}

\begin{tabbing}
Titlexxxxxxxxxxxxxxx \= \kill
\underline{TYPE} \> {\bf 		Action} \\
\end{tabbing}

\begin{tabbing}
Titlexxxxxxxxxxxxxxx \= \kill
\underline{SYNTAX} \> {\bf 		``sort scantime"} \\
\> {\bf 		``sort procdate"} \\
\> {\bf 		``sort snr"} \\
\> {\bf 		``sort length"} \\
\> {\bf 		``sort baseline"} \\
\> {\bf 		``sort frequency"} \\
\> {\bf 		``sort sourcename"} \\
\> {\bf 		``sort qcode"} \\
\end{tabbing}

\underline{DESCRIPTION}
\begin{list}{}{\setlength{\leftmargin}{0.5in}
     \setlength{\rightmargin}{0in}}
\item
Sorts the data in memory according the the value of the
field specified in the command.  The sort is stable, in that
entries in the database which compare equal will retain
their original sort order.  Thus any combination of sort
key priorities can be applied by repeated execution of ``sort".
\item
The only consequence of sorting the data is that the output
of the ``write" command will be sorted.  The various ``aedit"
commands do not care whether the data is sorted or not, and in
fact operate on unsorted data whether sort has been executed or
not.  Flagging and unflagging data does not affect the sort
order ... sorting is done on all the data.
\item
Note that reading in additional data destroys any sort order.  The
effect is the same as issuing the ``unsort" command, in that
the sort order information for the original dataset is explicitly
discarded, and the original sort order restored.
\end{list}
\vspace{.2in}

\begin{tabbing}
Titlexxxxxxxxxxxxxxx \= \kill
\underline{COMMAND} \> {\bf 	sources} \\
\end{tabbing}

\begin{tabbing}
Titlexxxxxxxxxxxxxxx \= \kill
\underline{TYPE} \> {\bf 		Data selection} \\
\end{tabbing}

\begin{tabbing}
Titlexxxxxxxxxxxxxxx \= \kill
\underline{SYNTAX} \> {\bf 		``sources 3C345, 3C273, OJ287"} \\
\end{tabbing}

\underline{DESCRIPTION}
\begin{list}{}{\setlength{\leftmargin}{0.5in}
     \setlength{\rightmargin}{0in}}
\item
Specifies a list of sources which will pass the filters in
edit inputs, read, and plot.  The source names must match
those in the data files exactly (including case), with the
exception of leading or trailing blanks.  Typing ``sources"
with no arguments removes any restriction on sources.
\end{list}
\vspace{.2in}

\begin{tabbing}
Titlexxxxxxxxxxxxxxx \= \kill
\underline{COMMAND} \> {\bf 	stations} \\
\end{tabbing}

\begin{tabbing}
Titlexxxxxxxxxxxxxxx \= \kill
\underline{TYPE} \> {\bf 		Data selection} \\
\end{tabbing}

\begin{tabbing}
Titlexxxxxxxxxxxxxxx \= \kill
\underline{SYNTAX} \> {\bf 		``stations ABC D EF"} \\
\end{tabbing}

\underline{DESCRIPTION}
\begin{list}{}{\setlength{\leftmargin}{0.5in}
     \setlength{\rightmargin}{0in}}
\item
Enters a list of allowed stations into the inputs.  All alphabetic
characters are accepted, in any order, lower or upper case, with or
without spaces or commas.  Duplicate characters are ignored.  Baselines
which involve stations not in this list will fail the filter tests
applied by edit inputs, read, and plot.  Typing ``stations" without
arguments removes any limitations on stations.
\end{list}
\vspace{.2in}

\begin{tabbing}
Titlexxxxxxxxxxxxxxx \= \kill
\underline{COMMAND} \> {\bf 	summary} \\
\end{tabbing}

\begin{tabbing}
Titlexxxxxxxxxxxxxxx \= \kill
\underline{TYPE} \> {\bf 		Action} \\
\end{tabbing}

\begin{tabbing}
Titlexxxxxxxxxxxxxxx \= \kill
\underline{SYNTAX} \> {\bf 		``summary" (no arguments)} \\
\end{tabbing}

\underline{DESCRIPTION}
\begin{list}{}{\setlength{\leftmargin}{0.5in}
     \setlength{\rightmargin}{0in}}
\item
Displays a summary of all unflagged data in memory on
the terminal.  An example is shown below:
\item
		SUMMARY OF UNFLAGGED DATA IN MEMORY
		-----------------------------------
\item
Total number of unflagged records = 556
\item
Earliest scan:       88-273-0931
Latest scan:         89-025-0905
Stations present:    KTYMHX
Baselines present:   KT TY MT MH MY YH MX YX XH 
Frequencies present: XS
SNR extrema:         0    393
Experiments present: 1963 1995
Sources present:     
	3C66A SN1986J 0224+67 NRAO150 3C66B 3C345 3C279 
	1921-293 2134+00 NRAO530 3C454.3 0106+013 0420-014 0528+134 
	0552+398 4C39.25 OJ287 0727-115 
Quality code summary:
	   A   B   C   D   E   F   0   1   2   3   4   5   6   7   8   9   ?
	   0   0   0  47   0   4  27   0   1   0   0   5   2  16  58 396   0
\item
There are 0 flagged records present
\item
\end{list}
\vspace{.2in}

\begin{tabbing}
Titlexxxxxxxxxxxxxxx \= \kill
\underline{COMMAND} \> {\bf 	timerange} \\
\end{tabbing}

\begin{tabbing}
Titlexxxxxxxxxxxxxxx \= \kill
\underline{TYPE} \> {\bf 		Data selection} \\
\end{tabbing}

\begin{tabbing}
Titlexxxxxxxxxxxxxxx \= \kill
\underline{SYNTAX} \> {\bf 		``timerange yyddd-hhmm yyddd-hhmm"} \\
\end{tabbing}

\underline{DESCRIPTION}
\begin{list}{}{\setlength{\leftmargin}{0.5in}
     \setlength{\rightmargin}{0in}}
\item
Sets the range of times outside which data will be rejected by
various filter-applying action commands (edit inputs, read).
Overrides self-scaling of the time axis on plots.  Typing
``timerange" without arguments removes any restriction on the
timerange.
\end{list}
\vspace{.2in}

\begin{tabbing}
Titlexxxxxxxxxxxxxxx \= \kill
\underline{COMMAND} \> {\bf 	type} \\
\end{tabbing}

\begin{tabbing}
Titlexxxxxxxxxxxxxxx \= \kill
\underline{TYPE} \> {\bf 		Data selection} \\
\end{tabbing}

\begin{tabbing}
Titlexxxxxxxxxxxxxxx \= \kill
\underline{SYNTAX} \> {\bf 		``type 0 1 2"} \\
\end{tabbing}

\underline{DESCRIPTION}
\begin{list}{}{\setlength{\leftmargin}{0.5in}
     \setlength{\rightmargin}{0in}}
\item
Sets the type data selection parameter.  In this version of
aedit, does nothing since only type 2 records can be read in.
\end{list}
\vspace{.2in}

\begin{tabbing}
Titlexxxxxxxxxxxxxxx \= \kill
\underline{COMMAND} \> {\bf 	unflag} \\
\end{tabbing}

\begin{tabbing}
Titlexxxxxxxxxxxxxxx \= \kill
\underline{TYPE} \> {\bf 		Action} \\
\end{tabbing}

\begin{tabbing}
Titlexxxxxxxxxxxxxxx \= \kill
\underline{SYNTAX} \> {\bf 		``unflag all"} \\
\> {\bf 		``unflag duplicates"} \\
\> {\bf 		``unflag cursor"} \\
\> {\bf 		``unflag qcodes"} \\
\> {\bf 		``unflag snrmin"} \\
\> {\bf 		``unflag timerange"} \\
\> {\bf 		``unflag stations"} \\
\> {\bf 		``unflag baselines"} \\
\> {\bf 		``unflag experiment"} \\
\> {\bf 		``unflag frequencies"} \\
\> {\bf 		``unflag type"} \\
\> {\bf 		``unflag sources"} \\
\> {\bf 		``unflag length"} \\
\> {\bf 		``unflag fraction"} \\
\> {\bf 		``unflag nfreq"} \\
\> {\bf 		``unflag parameter"} \\
\end{tabbing}

\underline{DESCRIPTION}
\begin{list}{}{\setlength{\leftmargin}{0.5in}
     \setlength{\rightmargin}{0in}}
\item
Unsets flag bits for all scans in memory, according to the
argument.  For example, ``unflag snrmin" unsets all flag bits
thoughout memory which were set with an ``edit inp" with
``snrmin" set higher than the scan snr value.  The combination
of ``edit" and ``unflag" allows great control over the flagging
status of the data in memory.  ``Unflag all" removes all flags
from the data simultaneously.
\end{list}
\vspace{.2in}

\begin{tabbing}
Titlexxxxxxxxxxxxxxx \= \kill
\underline{COMMAND} \> {\bf 	unsort} \\
\end{tabbing}

\begin{tabbing}
Titlexxxxxxxxxxxxxxx \= \kill
\underline{TYPE} \> {\bf 		Action} \\
\end{tabbing}

\begin{tabbing}
Titlexxxxxxxxxxxxxxx \= \kill
\underline{SYNTAX} \> {\bf 		``unsort"} \\
\end{tabbing}

\underline{DESCRIPTION}
\begin{list}{}{\setlength{\leftmargin}{0.5in}
     \setlength{\rightmargin}{0in}}
\item
Restores the original sort order of the data (i.e. the
order in which the data was read in).
\end{list}
\vspace{.2in}

\begin{tabbing}
Titlexxxxxxxxxxxxxxx \= \kill
\underline{COMMAND} \> {\bf 	write} \\
\end{tabbing}

\begin{tabbing}
Titlexxxxxxxxxxxxxxx \= \kill
\underline{TYPE} \> {\bf 		Action} \\
\end{tabbing}

\begin{tabbing}
Titlexxxxxxxxxxxxxxx \= \kill
\underline{SYNTAX} \> {\bf 		``write filename"} \\
\end{tabbing}

\underline{DESCRIPTION}
\begin{list}{}{\setlength{\leftmargin}{0.5in}
     \setlength{\rightmargin}{0in}}
\item
Writes all unflagged data in memory out to the filename
specified in the argument.  The data are written out
according to the current sort order (as determined by
execution of the ``sort" command).  If the data is not
sorted, the output order is the same as the order in
which the data was read.
\item
The input filters are ignored on ``write".
The way to write out selected data is to set the input
filters and then run ``edit inputs", before using ``write".
\end{list}
\vspace{.2in}

\begin{tabbing}
Titlexxxxxxxxxxxxxxx \= \kill
\underline{COMMAND} \> {\bf 	zoom} \\
\end{tabbing}

\begin{tabbing}
Titlexxxxxxxxxxxxxxx \= \kill
\underline{TYPE} \> {\bf 		Action} \\
\end{tabbing}

\begin{tabbing}
Titlexxxxxxxxxxxxxxx \= \kill
\underline{SYNTAX} \> {\bf 		``zoom" (no arguments)} \\
\end{tabbing}

\underline{DESCRIPTION}
\begin{list}{}{\setlength{\leftmargin}{0.5in}
     \setlength{\rightmargin}{0in}}
\item
Enables the cursor on an active plot on an interactive
graphics device.  The user selects a point by positioning
the cursor and typing any character except `x' or `X', and 
the program displays detailed information about that point 
on the terminal.  At present, the only information displayed 
is that resident in memory (i.e. A-file information), but in 
the future, the program will have the option of constructing
a data file name and displaying the FRNGE plot on screen.
\item
``Zoom" is terminated by typing an `x' or `X'.
\end{list}
\vspace{.2in}

