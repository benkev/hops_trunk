
\begin{center}
\LARGE\bf
\underline{AEDIT version 4.1}

\Large\rm
A program for editing, manipulating, visualizing and summarizing
MkIII/IV A-file format data
\vspace{.4in}

by \\
Colin Lonsdale \\
Haystack Observatory \\
\vspace{.3in}

\em September 1994

\begin{center}
\large\bf
\underline{New in 4.1}
\end{center}

Previous to this release of aedit, there were three types of A-file lines, 
corresponding to the three types of binary data files, root (type 0), 
corel (type 1) and fringe (type 2).  I have now added two new types of A-file 
lines.  They are type 3 (closure triangle) and type 4 (closure quad) lines, 
and they can be formed only by combining multiple type 2 (fringe) lines.  In 
this release of aedit I have implemented full support for type 3 lines only.  
Type 4 support will be added later at lower priority.  Closure triangle data 
can now be read, written, filtered, plotted, and interactively edited in much 
the same way as type 2 data.  You can get closure triangle data into aedit 
two ways.  First, you can read it in from a disk file using the standard read 
command.  Alternatively, you can read in some type 2, baseline data, and use 
the new close command to compute closure triangle data from the type 2 data.  
Either way, the closure data are fully plottable and manipulable.  It is 
important to note that aedit is (and should remain) the only HOPS program 
capable of generating closure data.  All closure triangle data in HOPS must 
ultimately originate from the close command in aedit operating on type 2 data, 
though it can be read, written and manipulated by aedit and other programs.


Because the data types are fundamentally different, type-3 closure triangle 
data and type-2 baseline data are stored internally in separate arrays.  They 
should also be stored on disk in separate files, so there is now a new twrite 
command (analogous to the existing write command), which writes out only closure 
triangle data.  If you really want to keep your baseline and closure data in one 
file, you can concatenate them and aedit will swallow the resulting file.  However, 
since other HOPS programs (such as average) will be expecting one or the other, 
and not both, it will probably be easier to keep them separate.

To accommodate closure triangle data, there are a couple of new filters you can 
set.  First, you can specify the closure triangles you want with the triangles 
command (station order is irrelevant), and second, you can specify a range of 
bispectral SNRs, using bsnrmin and bsnrmax.  The bispectral SNR calculated by 
the close command is the theoretical one based on the individual baseline SNRs.  
If the data come from a file instead, the bispectrum SNR may have been calculated 
differently (e.g. by program average).  Certain filters do not apply to certain data 
types, and are simply ignored.  For example, saying triangles ABC; edit inputs will 
have no effect on baseline (types 1 and 2) data.  All of the (relevant) filters 
familiar to you for baseline data, such as stations, quality code and so on work 
for closure triangle data.

There are some new options to the edit command which permit you to control the 
consistency of related baseline and triangle datasets.  The edit close triangles 
command removes triangle records which do not have all three legs of the triangle 
present in the baseline dataset.  The edit close baselines command, conversely, 
removes all baseline records which do not participate in at least one valid 
triangle record.  Just typing edit close does both operations, and ensures a 
fully consistent set of baselines and triangles.  The unflag command has been 
enhanced to accommodate the new types of flagging made possible by these filtering 
and editing changes.  Aedit makes no attempt to detect triangle redundancies 
(i.e. it will generate and/or accept N(N-1)(N-2)/6 triangles for N stations).  
While this may result in excessive amounts of closure triangle data for large 
experiments, it is necessary in order to preserve information in the low 
SNR case (where the triangles are not fully redundant), and besides simplifies 
what is already some quite convoluted logic for handling arbitrary input closure 
data files.

There have been a number of minor enhancements to the way various commands function.  
In all cases, these changes should cause no problems for existing aedit users.



\begin{center}
\large\bf
\underline{New in 4.0}
\end{center}

There have been many enhancements in ``aedit" since the release 3.0
in March 1992, some of which are major.  Below is a list of the most 
important enhancements.  The detailed documentation of the individual 
commands covers the numerous other, more minor changes.

\begin{enumerate}
\item Full support is now available for all three types of data record,
0, 1 and 2.  This applies to filtering, sorting, and other operations.
Two new ``edit" options are now available which remove data records
which lack parent or child data records (e.g. if a root record has no
associated corel or fringe records in the database, it can be edited
out with the command ``edit parents").  The ``family" sort option causes
data records to be grouped by root family when writing the data to disk
(normally, all type 2 records are written, then all type 1's, then all
type 0's).
\item Aedit now has significant experiment summary capabilities.  The
command ``ccread" causes a specified correlator control file to be read
into memory, after which one may invoke the ``psplot" command.  This is
a highly interactive experiment status browser and data editor, which
uses color-coded rectangles to represent data records.  You can tag or
edit data based on quality code, scantime or baseline, or by clicking on
individual data cells, and you can of course pop up fringe plots by 
clicking on any cell.  The display is intended to visually assist 
diagnosis of experiment-wide problems, among other things.
\item In order to circumvent the problem of limited numbers of parameters
in the standard A-file format, ``aedit" now supports parameter extraction
directly from the type-2 data files on disk.  Depending on circumstances,
you can now extract, plot, and write out any of a wide range of parameters.
You can also filter the data on the basis of the values of extracted parameters.
Phasecal information is now accessible only through the extracted
parameters.  The relevant commands are ``parameter", ``pwrite", ``plist"
and ``prange".
\item Plotting within ``aedit" has been improved significantly.  It is now
possible to plot any quantity, including extracted parameters if present,
against any other quantity (provided it makes rudimentary sense .. you
cannot plot closure rate versus baseline SNR, for example).  Specification
of the output device has been simplified for a subset of devices.  All
plots remain fully active, in the sense that on appropriate devices you can
edit data with the interactive cursor, or pop up fringe plots on Xwindow
devices.  New commands of relevance include ``xscale", ``yscale", ``reference"
and ``remote".
\item Along with all other MkIV programs, ``aedit" supports the new version 2
A-file format.  This format features a somewhat more useful mix of quantities,
with additional precision in many cases, and is to be preferred where
possible.  ``Aedit" can write the data out in any A-file format (see the
command ``outversion").
\end{enumerate}

\begin{center}
\Large\bf
\underline{GENERAL DESCRIPTION OF AEDIT}
\end{center}

The program ``aedit" is a general purpose A-file manipulation
program.  The information present in one or more A-files may
be plotted, filtered, sorted and edited in a variety of ways, 
before being written out in the form of a new A-file.  The user
interface to the program is presently implemented only for
ASCII terminals, but many functions of the program interact
with the user via a graphical interface.

Commands are given to aedit by keyboard, and a full minimum
match capability is supported for all aedit names.  Multiple
commands are allowed on one input line, the only requirement
being that commands are separated by the semicolon ``;" 
character.  Commands typically consist of the command name
followed by 0 or more arguments.  The arguments are separated
from the command name and each other by either spaces or
commas.  Aedit can handle long lines, but it is of course bad
practice to wrap lines on terminals in general.  Upwards of
about 250 characters may start to cause problems even for aedit.
In general, aedit is not case sensitive.  Case sensitivity
is needed for UNIX filenames and for certain quantities from 
A-files (station codes, frequency codes, source names).

Aedit uses the concept of inputs.  That is, you set up
certain variables in the program that determine how the ``action"
commands will behave.  Most of the commands that aedit
understands are of the input-setting variety.  Many are quite
particular about syntax, and will complain if the user types
nonsense (e.g. timerange).  At any time, the current state of
the input parameters can be listed on the screen with the
command ``inputs".

When aedit reads data from an A-file, it parses the ascii
information and stores it as binary data in memory.  This allows
very rapid manipulation of the data once read in, with
seemingly complex tasks appearing to be instantaneous.  There
is a flag field associated with each A-file line in memory,
and these flags are manipulated by the edit and unflag commands.
A full description of the data currently in memory can be
obtained with the ``summary" command.  This is essential when
deciding on plotting and editing options.  The ``write" command
ignores flagged data, permitting aedit to be used as a simple
and efficient filtering program.

A command ``run" is available, which provides a flexible and 
general command file capability.  Nesting of command files to a 
depth of 10 is allowed.  The ``run" command executes in batch mode, 
and cursor operations are therefore disabled.

Aedit features a shell escape.  By starting an input line with the
character ``!", you can access standard UNIX commands outside of
aedit.  You can escape to a complete new shell by typing ``!csh" or
``!sh", and when you have finished, return to aedit where you left
off by typing cntl-D.  This feature is useful for spooling plot
files to a printer, preparing run files, running ``alist" to prepare
new data for aedit, and any other tasks that you wish to perform
without terminating the aedit session.

Plotting is implemented by using the PGPLOT package from CalTech.
The output device may be specified with the ``device" command, or
you may leave ``device" at the default value (``?"), which will
cause PGPLOT to query you for a device at the time of plotting.
Your response will then be automatically entered into the ``device"
input.  A list of available device types can be obtained by responding
with a query.  For more information of devices, see ``help device"

Aedit comes with full on-line help.  In general, the syntax is
``help command", but just ``help" will work.

The command line for aedit is ``aedit [-x] [-r filename] [-f filename]",
where ``-x",``-r", and ``-f" are optional.  The ``-x" option means start
up the xwindow interface (not yet supported).  The ``-r" option means
execute the specified run file on startup, and must be immediately
followed by the name of a file containing valid aedit commands.  The ``-f" 
option means ``read this(ese) data file(s) on startup", and must be 
immediately followed by a standard, wildcardable UNIX filename specifier 
or specifiers.  In this way, you can read many files at once into aedit
without going through a laborious one-at-a-time ``read" cycle within
the program. If specified, the ``-f" flag must be the last flag.

Below is a list of all current aedit commands:

\begin{verbatim}
Action commands:
----------------
batch        clear       close       edit         exit
fplot        help        inputs      nobatch      parameter
plist        plot        pwrite      read         run
setyear      sort        summary     twrite       unflag
unsort       write       zoom


Plot control commands:
----------------------
grid         xscale      yscale      axis         mode
reference    remote


Data selection commands:
------------------------
baselines    bsnrmin     bsnrmax     experiment  fraction
frequencies  length      nfreq       prange      procrange
qcodes       snrmax      snrmin      sources     stations
timerange    triangles   type


Experiment overview commands/parameters
---------------------------------------
schedread    psplot      psfile


IO control commands:
--------------------
device       outversion

\end{verbatim}

For further information, see the individual help files for the
above commands.

\newpage
