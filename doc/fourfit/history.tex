In the late 70Õs Alan Rogers developed a program called FRNGE derived from the original VLBI2 fringe-fitting program. It was written in fortran and designed to be efficient on an HP-21MX (later renamed to HP-1000) minicomputer. This computer had only 32K x 16 bit words, and was extremely slow by 21st century standards. Just performing the fringe-fitting on small scans in several minutes time was a real tour de force.

As both hardware and software technology improved it became apparent that FRNGE would need to be rewritten, and so fourfit was created by Colin Lonsdale, Roger Cappallo, and Cris Niell in the early 90Õs. The basic algorithms were adopted directly from FRNGE, while the I/O, control architecture, and plotting were heavily modified. The programming language was changed from fortran to C.
