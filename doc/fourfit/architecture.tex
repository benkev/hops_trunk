\subsection{control files}
Internally, the control files are implemented as a chained list of
structures, which are created and appended in the order that the control
file information
is encountered by the program (i.e. default control file first, then the
main control file, then any \textbf{set} commands.) The control block structures
each have four selectors, which control whether or not a particular
block will be applied to data. The selectors are (baseline, source, frequency
group, and time range), with wildcard matching allowed. A fifth 
"pseudo-selector" is a single-character station code such as K. It is expanded
internally into two identical parameter blocks, one matching baseline K? and the
next matching ?K. In addition to the selectors there are many parameters
that can be applied to the data, such as filtering parameters, calibration
corrections, or search ranges.

The control file is parsed by a finite-state machine, which can at times
react to errant input files so as to produce syntax error messages 
that are a little hard to interpret. The offending input line is always
printed, though, to help the user zero in on the problem.

The principal routines used in the control file parsing are:
\begin{itemize}
\item create\_fsm -- sets up a structure with the finite-state-machine information
containing the semantics of the control file commands
\item parse\_cmdline -- parses the command line options
\item parse\_control\_file -- high level routine that reads the control file, sets up
the fsm, uses lex, and invokes the parser
\item lex -- does a lexical analysis, changing keywords into tokenized information
\item parser -- does the heavy lifting of applying the fsm semantics to the tokenized
output of lex, in order to create c\_blocks
\end{itemize}

\subsection{fringe search}
The fringe search in \textit{fourfit} is a two-stage process:

\begin{enumerate}
\item Search over a large 3-dimensional grid of (singleband delay,
multiband delay, delay-rate) to find the approximate location of
the maximum of the delay resolution function.

\item Form a 5x5x5 cube of correlation amplitude 
values centered on the location found in the grid search, and
perform a 3 dimensional interpolation to find the peak value.
Note: this section describes the \textit{simultaneous} interpolation,
not the now-deprecated \textit{iterative} interpolation.
\end{enumerate}

\subsubsection{grid search}
A search is made through a large number of grid points
in a 3-dimensional volume with axes of singleband delay,
multiband delay, and delay-rate. The function being maximized
is the coherent sum of all of the correlator output
complex visibility points over frequency and time. The grid points are
evenly spaced, which allows the search to be done efficiently via
FFT's. For example, along the delay-rate axis the transform
of the complex phasors converts from a function of time
to a function of fringe frequency. The best fit can be found
by inspecting the fringe frequency values to find the maximum.
Similar conjugate searches in the delay domains are done: the
cross-power spectral points within the channels determine the
singleband delay peak, and the phases across the different
channels determine the multiband delay peak.
\subsubsection{interpolation}
Once the location of the maximum grid point is known, then an
interpolation step refines the values in the neighborhood of
that (singleband delay, multiband delay, delay-rate) triple.
In the interpolation step a direct summation of the complex visibility
values is performed, by rotating the data by a complex phase 
factor (see next section). This factor incorporates the effect of the offsets
along each of the three axes of a 5x5x5 cube, which is centered
on the values (sbd, mbd, dr) from the grid search.
After the cube is formed, the 3 coordinates of the maximum
of the visibility magnitude is found via iterative application
of a simultaneous 3-D interpolation. The iteration step is implemented
by searching for the maximum of an interpolated 11x11x11 cube
of visibility magnitudes. Each iteration has a 11x11x11 cube that
is a factor of 5 smaller, until all 3 coordinate spacings are
less than 0.0001 of the original 5x5x5 spacing.
\subsubsection{complex phase factor}
In the interpolation of the best-fit (mbd, sbd, dr) values and also
in the plotting of the data as fit by the program, there is
a complex rotation applied to the data. This factor has
the form $e^{-i 2 \pi \theta}$, where $\theta$ is given by
\begin{equation}
        \theta = f \Delta t \dot{\tau} + (f - f_{ref}) \tau_{mb} + b \tau_{sb}
\end{equation}
where:

\begin{tabular}{ll}
    $f$         & frequency of a data point \\
    $f_{ref}$   & reference frequency \\
    $\Delta t$  & offset in time of data from ref. time \\
    $\dot{\tau}$ & residual delay rate \\
    $\tau_{mb}$ & residual multiband delay \\
    $\tau_{sb}$ & residual singleband delay \\
    $b$         & (0.25, 0, -0.25) for (usb, dsb, lsb) data \\
\end{tabular}
\subsubsection{program structure}
The overall program structure can be seen in the calling
graphs of Appendix \ref{app:callinggraphs}.
