\subsection{Selector keywords}
\begin{tabular}{ll}
    \textbf{Keyword} & \textbf{Value} \\
    station         &  1 character \\
    baseline        &  2 characters \\
    source          &  string of 1-8 chars \\
    f\_group        &  1 character \\
    scan            &  UT-epoch (special format), or: \\
                    &  $<$ UT-epoch \\
                    &  $>$ UT-epoch \\
                    &  UT-epoch1 to UT-epoch2  (inclusive time range) \\
\end{tabular}
Notes:
\begin{enumerate}

\item The wildcard character '?' can be used for any or all characters
within the baseline, the source name, or the frequency group. So long as
the other, non-wildcarded characters all match the scan will be selected.

\item The station selector keyword is provided for the user's convenience,
but internally within the program there is no station keyword. Internally,
each use of the station keyword results in two matching control blocks,
so that if station Z (commands) is equivalent to if baseline Z? (commands)
followed by another block with if baseline ?Z (same commands). This subtlety 
is not normally apparent to the user, but could arise, e.g., if one tried
some compound condition involving both a station and a baseline, such as:

if station Q and baseline QC \\

Such combinations are probably best avoided.

\end{enumerate}



\subsection{Syntactic keywords}
\begin{tabular}{ll}
   if & \\
   else &   (not yet implemented) \\
   and & \\
   or & \\
   not & \\
   ( )  &    (not yet implemented) \\
   $< >$ & \\
   to & \\
   ? & \\
\end{tabular}

\subsection{Action keywords}
\begin{tabular}{ll}
    \textbf{Keyword} & \textbf{Value} \\
   adhoc\_amp        & float \\
   adhoc\_file       & string \\
   adhoc\_file\_chans & string \\
   adhoc\_period     & float \\
   adhoc\_phase      & `sinewave', `polynomial', or `file' \\
   adhoc\_poly       & $\leq 6$ floats/integers (mixture OK) \\
   adhoc\_tref       & float \\
   dc\_block         & `true' or `false' (def: false) \\
   dec\_offset       & float \\
   delay\_offs       & n char string, followed by n floats \\
   dr\_win           & 2 floats \\
   est\_pc\_manual   & int \\
   fmatch\_bw\_pct   &  float \\
   freqs            & n chars \\
   gates            & n char string, followed by 2n floats \\
   gen\_cf\_record  & `true' or `false' (def: false) \\
   interpolator     & 'iterate' or 'simul' (def: simul) \\
   ionosphere       & float \\
   ion\_npts        &  int (def: 0)\\
   ion\_smooth      & `true' or `false' (def: false) \\
   ion\_win         &  2 floats \\
   lsb\_offset      &  float \\
   mb\_win          &  2 floats \\
   mbd\_anchor      &  `sbd' or `model' (def: model) \\
   notches          & 2n floats \\
   optimize\_closure & `true' or `false' (def: false) \\
   passband         & 2 floats \\
   period           & int \\
   pc\_amp\_hcode     & float \\
   pc\_delay\_l       & float \\
   pc\_delay\_r       & float \\
   pc\_delay\_x       & float \\
   pc\_delay\_y       & float \\
   vbp\_correct       & `true' or `false' (def: false) \\
   vbp\_fit           & `true' or `false' (def: false) \\
   vbp\_coeffs        & 5 floats/integers (def: 0.0) \\
   vbp\_file          & string \\
\end{tabular}
\begin{tabular}{ll}
   pc\_phases        & n char string, followed by n floats \\
   pc\_phases\_l     &  n char string, followed by n floats \\
   pc\_phases\_r     &  n char string, followed by n floats \\
   pc\_phases\_x     &  n char string, followed by n floats \\
   pc\_phases\_y     &  n char string, followed by n floats \\
   pc\_freqs         & n char string, followed by n floats \\
   pc\_mode          & `normal', `ap\_by\_ap', `manual', or 'multitone' \\
   pc\_period        & int \\
   pc\_tonemask      & n char string, followed by n ints \\
   ra\_offset        & float \\
   ref\_freq         & float \\
   samplers         & int, followed by up to 8 strings \\
   sampler\_delay\_l &  up to 8 floats \\
   sampler\_delay\_r &  up to 8 floats \\
   sampler\_delay\_x &  up to 8 floats \\
   sampler\_delay\_y &  up to 8 floats \\
   sb\_win         &   2 floats \\
   skip            &  `true' or `false' \\
   start           &  integer \\
   station\_delay  &   float \\
   stop            &  integer   \\
   switched        &  `scan\_start' or `each\_minute' \\
   t\_cohere       &   float \\
   use\_samples    &   `true' or `false' \\
   weak\_channel   &   float \\
\end{tabular}

\vspace{0.3in}
\textbf{\textit{-- deprecated commands -- for backward mk4 compatibility only:}} \\
\vspace{0.3in}
\fbox{\begin{tabular}{ll}
   index      &  n ints - alternate method used to select data channels, based on the \\
              &  original corel index number. Not as well supported as freqs, \\
              &  and is currently being phased out. \\
   max\_parity  &      float (tape only) \\
   x\_crc       &      `keep' or `discard' (tape only) \\
   x\_slip\_sync &      `keep', `discard', or an integer (tape only) \\
   y\_crc        &     `keep' or `discard' (tape only) \\
   y\_slip\_sync &      `keep', `discard', or an integer (tape only) \\
\end{tabular}}

\subsection{Keyword semantics}

\subsubsection{Scan Selection}

\begin{itemize}
\item[]\textbf{skip} -- if this is set to true in the body of an if\_block, then
                 any scans matching the if conditions will be skipped. 
                 Note: as of 99.2.19 \textit{fourfit} will not properly skip data 
                 if f\_group is specified.
\end{itemize}

\subsubsection{Filtering -- \textit{determine whether or not each AP is accepted}}

\begin{itemize}
\item[]\textbf{freqs}        controls which frequency channels get included in the fit.
                 The letters a-p correspond to the order that the frequencies
                 appear in the root file (assuming 16 channels). With no
                 suffix, DSB is implied, if both sidebands are present.
                 A plus suffix denotes USB, a minus is used for LSB.
                 After 26 channels, the uppercase alphabet is used,
                 then 10 digits, finally '\$' and '\%' (i.e. 64 channels).
\item[]\textbf{start}        start time for data to be included
\item[]\textbf{stop}         stop time for data to be included \\
                 Arguments of start and stop are integers with an optional
                 minus sign. A positive integer is interpreted as an
                 absolute time in seconds past the hour (of the scan
                 start time). When a minus sign precedes the start time
                 it is considered to be a time relative to, and later
                 than, the scheduled scan start. Similarly, a negative
                 stop time precedes the scheduled scan stop time, by
                 the indicated number of seconds.
\item[]\textbf{switched}     turns on (frequency) switched mode, which discards some AP's
                 and keeps others, depending on a gating waveform
\item[]\textbf{period}       period in seconds of the gating waveform
\item[]\textbf{gates}        for each freq. channel, the starting delay and duration, in
                 seconds, of the gating waveform
\item[]\textbf{passband}     lower and upper bounds (in MHz) of the spectral passband of
                 data to be accepted, specified as RF frequencies. If the
                 lower bound is greater than the upper bound, the range
                 wraps around -- allowing a band in the middle to be
                 excluded.
\item[]\textbf{notches}     a list of non-overlapping [lower,upper] bounds pairs (in MHz)
                 to exclude from the spectral passband (passband may be
                 applied prior to removal of these notches). Note that the
                 amplitude modification calculus isn't sufficiently
                 sophisticated to detect overlaps of passband and notches,
                 so be sure to keep your surgeries disjoint.  (A large number of notches
                 is supported and you will get a complaint if you exceed it.)
                 As with passband, the spectral data is rescaled so that
                 the amplitude observable is in some sense preserved.
\item[]\textbf{dc\_block}     if set to true, zero out lowest cross-power spectral
                 channel; useful for suppressing DC bias
\end{itemize}

\vspace{0.3in}
\textbf{\textit{-- deprecated commands -- for backward mk4 compatibility only:}} \\
\vspace{0.3in}
\fbox{\begin{tabular}{ll}
   max\_parity   & maximum allowable fraction of bytes in error \\
   x\_slip\_sync &  maximum number of frames w/ re-sync in reference station \\
   y\_slip\_sync &  (same for remote station) \\
   x\_crc        & either keep or discard AP if reference time has CRC error \\
   y\_crc        &  (same for remote station) \\
\end{tabular}}

\subsubsection{search -- \textit{control the fringe-searching process}}

\begin{itemize}
\item[]\textbf{sb\_win}       single band delay search window bounds, in us
\item[]\textbf{mb\_win}       multiband delay search window bounds; if the upper
                 multiband delay bound (2nd number) is less than the lower bound
                 (1st number), then \textit{fourfit} performs a "wrap-around" search, in
                 order to handle the case of a delay near to the multiband
                 (semi-) ambiguity.
\item[]\textbf{dr\_win}       delay-rate search window bounds, in us/s 
\item[]\textbf{ion\_npts}     number of evaluation points in ionospheric coarse search.
        Either 0 (the default value) or 1 disables the ionosphere model.
\item[]\textbf{ion\_smooth}   iff true, use alternative ion search strategy in which
        tec tabular points are smoothed mid-search
\item[]\textbf{ion\_win}      ionospheric coarse search window in TEC units
\item[]\textbf{ra\_offset}    apply right asc.offset (asec) to re-center search windows
\item[]\textbf{dec\_offset}   apply declination offset (asec) to re-center search windows
\item[]\textbf{interpolator} selects method of fit interpolation. Classically, an
                 iterative search has been done over sbd, mbd, drate, one
                 dimension at a time, for 3 cycles. The simultaneous mode
                 constructs a 5x5x5 cube of data points and does a 3D
                 quintic interpolation.
\end{itemize}


\subsubsection{corrections -- \textit{apply corrections to the data, either before or after fit}}

\begin{itemize}
\item[]\textbf{pc\_mode}  specify phase\_cal mode:
              - normal (model linear in time is extracted from the data)
              - manual (specified totally by the user) 
              - ap\_by\_ap (phase cal is extracted independently for each AP)
                DEPRECATED: use normal or manual with pc\_period 1 or more
              - multitone (all tones in band are coherently fit, and phase 
                is extrapolated to the center of the band).
\item[]\textbf{pc\_phases} phase-cal phases in deg, for each of the listed freq channels;
             these offset phases are added to the underlying model, as
             specified by pc\_mode, above. If 2 polarizations are present,
             the same values are applied to both pols.
\item[]\textbf{pc\_phases\_l} specified in same manner as pc\_phases, but the tone phases
             so specified are applied only to the first pol (L, X, or H)
\item[]\textbf{pc\_phases\_r} specified in same manner as pc\_phases, but the tone phases
             so specified are applied only to the second pol (R, Y, or V)
\item[]\textbf{pc\_phases\_x} synonym for pc\_phases\_l (see)
\item[]\textbf{pc\_phases\_y} synonym for pc\_phases\_r (see)
\item[]\textbf{pc\_freqs}  phase cal tone frequencies in KHz, for each of the listed
             freq channels iff not in range -64..64. Inside of this
             range, the value is interpreted as a tone number, with 1 being
             the 1st USB tone, 2 being the 2nd USB tone, etc. Negative
             tone numbers are used for LSB tones.
\item[]\textbf{pc\_period} in multitone mode (only), the phase can be estimated
             and applied over each pc\_period ap's, thus removing slopes
             or other drifts in pcal (default is 5)
\item[]\textbf{pc\_tonemask} - in multitone mode (only):
             The values for pc\_tonemask form a bit-masked map of which 
             tones to *exclude* for this frequency channel. Thus 1 
             excludes the lowest tone, 2 the next lower tone, 4 the 3rd 
             lowest tone, etc. A value of 5, for example, would exclude 
             the lowest and the 3rd lowest tones (perhaps 10 KHz and 2.01
             MHz).  
\item[]\textbf{pc\_delay\_l} a time value in ns representing the difference between the
\item[]\textbf{pc\_delay\_r} the travel time from the feed phase center to the pcal
             injection point, minus the the travel time from the pcal
             pulse generator to the injection point. It is specified
             separately for the two polarization senses.
\item[]\textbf{pc\_delay\_x} 
\item[]\textbf{pc\_delay\_y} synonyms for pc\_delay\_l and pc\_delay\_r (see)
\item[]\textbf{lsb\_offset} additive phase in degrees, for the LSB relative to the USB;
             often necessary when correlating VLBA data against Mk 3
\item[]\textbf{ref\_freq}  specifies a frequency in MHz at which the phase delay
             is determined (default is total LO of first frequency)
\item[]\textbf{adhoc\_phase} specify mode of ad hoc phase corrections. No corrections
             are made if this isn't present, or is set to false.
\item[]\textbf{adhoc\_period}  For ad hoc sinewave model; the period in integer seconds.
\item[]\textbf{adhoc\_amp}      "   "  "     "       "    amplitude in degrees of phase.
\item[]\textbf{adhoc\_tref}    For both ad hoc phase models; the reference time in seconds
             past the most recent hour.
\item[]\textbf{adhoc\_poly} For the ad hoc phase polynomial model; From 1-6 coefficients
             describing a power-series model in time. ($deg/sec^n$)
\item[]\textbf{adhoc\_file} Name of the file containing phases in the ad hoc file mode.
\item[]\textbf{adhoc\_file\_chans} String of channel labels for phases (columns) in the
             ad hoc file.
\item[]\textbf{use\_samples}   If true, use the sampler statistics (aka state counts) to
             normalize the raw correlation sums to the equivalent analog
             correlation. This is specific to the mk4 hardware correlator, as the state
             information is not otherwise available.
\item[]\textbf{ionosphere} specified per station, in TEC ($10^{16}el/m^2$) units
\item[]\textbf{t\_cohere}  coherence time used in fringe fit (default is infinite). The
             principal use case for t\_cohere is to cope with limited atmospheric coherence
             at high frequencies. \textit{fourfit} will add together multiple delay-rate channels
             in an incoherent manner, effective smoothing the correlation function over
             delay rate.
\item[]\textbf{delay\_offs} delay offsets (ns) to be applied to each of the listed freq
             channels. This correction is made prefit, similar to pcal.
\item[]\textbf{samplers}  the number of samplers, followed by the freq channel 
             identifiers of channels that share a common sampler are
             grouped together in strings. In multitone mode only, the
             averaged tone-derived differential delays are applied to 
             all channels sharing a sampler. In VEX2 the plan is to get this
             equipment-interconnection information from the VEX file.
\item[]\textbf{optimize\_closure} modifies fine fringe search so as to minimize the
             contribution of non-closing delay errors to the closure phase;
             can result in poorer single-baseline fits
\item[]\textbf{station\_delay} a priori guess at the delay of the pcal path, from maser
             to the digitizers (ns). Recommended to use 
             sampler\_delay\_l/r/x/y instead.
\item[]\textbf{mbd\_anchor} controls the basis for choosing the mbd ambiguity when
             forming the total. If 'sbd' then the ambiguity closest to
             the singleband delay is chosen; if 'model' then the
             ambiguity closest to the a priori model is chosen.
\item[]\textbf{sampler\_delay\_l} indexed by sampler, it is the center of the window
             in which the pcal delay ambiguity will be resolved. Just like
             the old station\_delay, but now broken down by sampler and
             polarization, since their cabling and filters will lead
             to different delays. For LCP, in units of ns.
\item[]\textbf{sampler\_delay\_r} same as the above, but for RCP, instead of LCP.
\item[]\textbf{sampler\_delay\_x} synonym for sampler\_delay\_l.
\item[]\textbf{sampler\_delay\_y} synonym for sampler\_delay\_r.
\item[]\textbf{vbp\_correct} if true, apply video bandpass correction.
\item[]\textbf{vbp\_fit} if true, fit for and print video bandpass correction.
\item[]\textbf{vbp\_coeffs} numerical coefficients for video bandpass correction.
             The algebraic model used is a cubic polynomial plus a 1/f term. If the
             five coefficients supplied are called $a$ through $e$, then the model is
             $\Delta \phi = a f^3 + b f^2 + c f + d + e/f$. The coefficients are in
             degrees of phase.
     \item[]\textbf{vbp\_file} (NYI) if specified, use file-based video bandpass correction 
        from given file (supersedes algebraic model correction).
\end{itemize}

\subsubsection{switches and other miscellaneous functionality}
\begin{itemize}
\item[]\textbf{fmatch\_bw\_pct} associate freqs < this percentage of bw together (25\% default)
\item[]\textbf{pc\_amp\_hcode}  generate an H code iff any pcal amps less than this (0.005 default)
\item[]\textbf{weak\_channel}  the ratio of single\_channel\_amp to coherent\_sum\_amp 
                 below which a G code is assigned to the scan (0.5 default)
\item[]\textbf{gen\_cf\_record} if true,
    saves the full control file in the fringe record
\item[]\textbf{est\_pc\_manual} if nonzero, estimates manual
    pc\_phase\_? and delay\_offs\_?
    values. A value >0/<0 estimates ref/rem station values.
    The magnitude is a mask of bits indicating what and how
    to make estimates.  0x01 phase, 0x02 median channel sbd,
    0x04 average sbd channel, 0x08 use total sbd value,
    0x10 use original per-channel sbd delay, 0x20 activates
    heuristics to discard outliers and 0x40 reports phase
    measurement as a pc\_phase\_offset\_? value.  Calculations
    are made with channels in the -f and -n range (inclusive).
    Conflicting commanding on delay options does nothing.
\item[]\textbf{chan\_ids} followed by a string of channel labels
        and an array of RF frequencies. Allows override of the default single character 
    channel labels. The labels are applied iff the corresponding RF freq matches to 0.01 MHz or better.
\end{itemize}

\subsubsection{other symbols}
\begin{tabular}{ll}
   ?       & wild card character - for single char within baseline, source or f\_group \\
   keep    & 32767 \\
   discard &     0 \\
   true    &     1 \\
   false   &     0 \\
\end{tabular}
   


\subsubsection{Special Formats}
   UT-epochs:  UT-epochs are expressed in the format ddd-hhmmss, where all 10
           characters are necessary, including leading 0's if
           appropriate.  This format will match that of a scan directory,
           if the UT-epoch that is being specified is an actual scan time.


\subsubsection{General Guidelines}
\begin{enumerate}
\item White space is ignored; i.e., multiple spaces and line feeds all
      collapse to a single space.
\item Multiple commands per line are fine.
\item Comments: anything from an asterisk through the end of the line is ignored.
\item Nested ifs are not allowed (or necessary). Nested parentheses in
      an if condition are fine (not yet implemented).
      As of 94.1.16, parentheses are not supported. The logical operators,
      in decreasing order of precedence are (not, and, or).
\item Wildcard "?" matches any single character for f\_group, station, or
      baseline, any string (of up to 8 characters) for source, and any
      time-value for scan.
\item Phase cal and delay offsets are treated station by station. If not
      in a "station context", then values are applied to remote stn only.
\item Only freqs that are chosen for both stations in a baseline are 
      present in the fit.
\item If multiple if-blocks match a particular passes' choice of baseline,
      f\_group, source, and scan criteria, then the later values assigned
      to each parameter overwrite the earlier ones.
\end{enumerate}

