\subsection{Introduction}

In the past, there were three different phase calibration modes that were employed in fourfit, with each having its own strengths and applicable usage cases. A brief summary of their characteristics is given in Table 1.  This memo describes a new mode, called multitone mode, which introduces new capabilities, as well as incorporating most of the functionality of the 3 previous modes.


mode
description
advantages
disadvantages
normal
any single tone extracted by corr. hardware
simplicity
poor snr due to ignored tones
manual
each channel's phase is manually specified
can be determined by peaking on a strong calibrator and used for a whole experiment; no thermal noise added
labor-intensive; requires calibrator scan; doesn't follow scan to scan variation
ap\_by\_ap
pcal phase for a single tone is extracted and applied independently for each AP
able to track a `wandering' phase
introduces extra noise into fringe phase; poorest snr
Table 1 Prior phase cal modes in fourfit


\subsection{Multitone Features}
\begin{itemize}
\item{multiple tones within each channel}: the coherent sum of all tones within the channel's bandpass is formed, a best-fit delay determined, the phase is evaluated at mid-band, and applied
\item{masking of tones by channel}: in order to avoid contamination of the phase calibration by strong interfering tones, it is possible to select for use any subset of the tones within the band, by way of a  excluded tone mask
\item{flexible averaging over time}: since phase corrections can vary over time within a scan, perhaps not even in a linear fashion, the multitone phasecal correction can be segmented over time. The integration period is user-adjustable so that the phase can be tracked with a minimum effect on the noise added into the interferometric fringes by pcal phase noise.
\item{additive phases by channel}: for flexibility it is sometimes useful to apply an extra additive phase, on a channel by channel basis
\end{itemize}

\subsection{Mathematical Underpinnings}
Let us define:

\begin{tabular}{ll}
   $N$       &  number of tones within the band \\
   $f_n$     &  frequency of the nth phase cal tone within the band (Hz) \\
   $f_c$     &  center frequency of the band (Hz) \\
   $\rho_n$  &  complex phasor of the nth tone, as measured \\
   $\tau_n$  &  best-fit delay (s) to tones across the band \\
   $P_n$     &  coherent sum at mid-band \\
\end{tabular}

To find the best-fit delay we take the complex phasors for all of the included tones
within the band, which due to their means of generation by a short pulse - all fall
nicely on a grid of equally spaced points in frequency, and find the peak (delay)
of their FFT. In order to increase accuracy, the phasors are first 0-padded, which
is equivalent to using a sinc interpolator on the delay function. Using the best-fit
delay, we then counter-rotate the individual phasors to the center frequency
(since we are correcting the mean channel phase) and form their sum:

    
\begin{equation}
        P_n = \sum_{n=0}^{N-1} \rho_n e^{-i2\pi(f_n-f_c)\tau_n}
\end{equation}
\vspace{0.3in}
The phase of this complex result, \textit{arg} ($P_n$), is used to correct the 
fringe phase of the channel.
 
\subsection{Syntax and Usage}

A fourfit user invokes and controls multitone mode by adding one or more lines 
to the fourfit control file. The stations to which a particular specification 
is applied are determined by the normal fourfit control file context rules. 
The applicable stations could be established, for example, by a construct 
of the form if <station> or if <baseline>, or the scope could be global 
(specified prior to any ifs). The relevant commands, all of which are optional, are

\begin{itemize}
        \item\textbf{pc\_mode multitone} command to specify multitone mode

        \item\textbf{pc\_period <period>} an integer specifying the averaging period in AP's 
                (default is 5 ap's). pcal ap's are typically 1 sec in duration. The optimal value
                to use depends on the pcal signal strength and the variation in time of the pcal
                signal. Smaller/shorter values add more thermal noise in the phase correction but track
                the phase well, larger/longer values add less Gaussian noise but are susceptible
                to phase drifts.

        \item\textbf{pc\_tonemask <chan\_specifier> <mask1> <mask2>} an integer tone exclusion mask
                is given for each of the specified channels; each bit within an exclusion mask
                corresponds to a tone within the channel; if set, bit 0 omits the 1st tone from 
                the fit, bit 1 omits the 2nd tone, and so on. This is useful for leaving out
                RFI-corrupted tones from the fit across the channel.

        \item\textbf{pc\_phases <chan\_specifier> <ch1\_phase> <ch2\_phase>} specifies, for each named channel,
                an additional phase to be added to the values normally extracted by multitone mode.
\end{itemize}


\subsection{Examples}

Example 1: comparison of normal and multitone phase cal modes when the tone 
sequence occurs at different points within different bands. When using normal
mode the correction phase depends on the placement of the tone that is being extracted,
due to the delay from injection point to where the waveform is sampled.





