This manual is intended to be useful for a varied audience, which will
include both users and developers. Hopefully it is organized enough
so that any specific user can find the information they need, although
of course the ultimate arbiter is the source code, which should be
consulted for further detail.

\subsection{Why fringe-fit?}
One might reasonably ask why fringe-fitting is even necessary, given that
the data have already been "correlated". The reason is that correlation
is done with a very specific model, which incorporates information about
source coordinates, station coordinates, epoch, frequency sequence, polar
motion, earth rotation, clock-settings, etc. Many of these parameters
contain significant errors, which will manifest themselves as non-zero
residual delay and phase, varying in time. It is the task of fringe-fitting
to remove as much of this residual signal as possible by estimating corrections
to the intermediate quantities of group delay and delay rate.

Once corrections to delay and rate are estimated, they can then be recombined
with the delay and rate predicted by the model, in order to yield a "total"
observable. This total is relatively insensitive to errors in the original
model. For example, if we define $O$ as the observed (true) value and $C$ 
as the computed model value, then the residual data value is $O-C$, and we 
have the relation $O = C + (O-C)$. Small changes to $C$ are canceled out by
compensating changes in $O-C$, at least over the region where changes
remain linear.

\subsection{Correlator windows}
Not only is it desirable to keep the model as close to reality as possible,
in order to stay within the linear region, but there are also some natural
limitations inherent in the correlation process, which cause a finite range
of possible residual values around the model. For example, a lag correlator
has a finite number of lags, and there is a maximum lag between the two 
datastreams, typically of order of 1 or 2 $\mu$s. The models must agree
with reality to this level, or the correlated signal will fall outside
the range of correlation. An FX correlator has similar limitations. If the
delay rate residual is too large, then the fringe phase may wrap around
a half or a full turn per accumulation period, causing loss of fringes or
weakened fringes, respectively. The natural multiband delay window is set 
by the spacing of the frequencies of the channels. It is the inverse of the
GCD of the all of the channel frequency differences.

For all three windows the natural window can be further restricted in
the post-processing. This is done in order to speed up the fringe search,
or possibly to restrict the range of a parameter based upon externally-
supplied information.
