To add a new (simple) parameter, one must modify the source
code in a number of routines. Even though there are a number
of places to change, the structure of the code ensures that
the changes are generally straight-forward.
The necessity of changing the routines marked as optional 
depends on the specifics of the parameter that is being added.
\begin{enumerate}
\item\textbf{control.h}
    most likely the paraameter value needs to be kept in struct c\_block

\item\textbf{parser.h}
    most likely you'll need to provide a new token for the parameter

\item\textbf{param\_struct.h} (optional)
    possibly add new parameters to this global structure

\item\textbf{init\_tokens.c}
    maps your tokens to the control file strings

\item\textbf{default\_cblock.c}
    set the default value

\item\textbf{nullify\_cblock.c}
    set to recognizible null value

\item\textbf{copy\_cblock\_parts.c}
    a line to copy it when non-null

\item\textbf{parser.c}
    appropriate fsm changes

\item\textbf{precorrect.c} (optional)
    changes prior to fringe search

\item\textbf{create\_fsm.c} (optional)
    if your change changes the fsm

\item\textbf{../../help/fourfit.doc}
    document what the new parameter does

\item\textbf{../../data/ff\_testdata/chk..sh} (optional)
    consider adding a test to make check in
    order to verify that what you changed
    does what you want (now and into the future).

\end{enumerate}
