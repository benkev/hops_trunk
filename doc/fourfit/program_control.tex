\subsection{\textbf command line}
The \textit{fourfit} command line has a number of options, which are
detailed in Appendix \ref{app:commandlineoptions}. 
Depending on the chosen options, each execution of \textit{fourfit}
could process any combination of:
\begin{itemize}
\item one or more scans
\item one or more baselines
\item one or more frequency bands
\item one or more polarization products
\end{itemize}
The most commonly used options include -t to invoke test mode,
which doesn't create any output files, -b to specify a particular
baseline to process, -p to pop up a fringe plot on the screen,
and -c to specify a control file name.


\subsection{\textbf control file}
Control files are the principal means by which \textit{fourfit} is usually controlled.
There is normally one control file created per experiment, with finer control
of processing provided within an experiment by using optional selectors,
such as specifying a time range or particular scans,
within the control file. Control file information can in turn come through
three different routes, in the order they are read:
\begin{enumerate}
\item there is the capability (not normally used) of having
a default control file. If it exists, it is accessed by
an environment variable called \$DEF\_CONTROL. It might prove useful
if the program defaults need to be changed on a regular basis.
\item a control file that may be referenced
through the -c option of the \textit{fourfit} command line
\item a convenience feature that allows immediate use of control file commands
on the run line (see \textbf{set} below)
\end{enumerate}
If the same parameters are set in more than one location it is the last
setting that is used.

\subsection{\textbf{set} mechanism}
As explained above, the final control file input to the program
is anything on the command line following \textbf{set}, if that is present.
Since it comes last, the information - if it matches the selector
criteria - overwrites any corresponding earlier information. For example,
the experiment control file might specify the singleband delay range over
which to search, but one could override it by appending 
\textbf{set sb\_win 0.56 0.56}, which would force it to use the value of 
0.56 $\mu$secs. This mechanism provides a convenient way to see the
effect of quick changes to parameters, since it avoids having to edit
the control file.
