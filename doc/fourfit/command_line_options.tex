
\textbf{Command line syntax:}

\texttt{fourfit [-a] [-b BB:F] [-c controlfile] [-d display device]
            [-f value] [-m value] [-n value] [-p] [-r afile] [-s naps]
            [-tux] [-P polar\_pair] [-T trefoffs] [-X] data file list 
            [set <control file syntax statements>]}

         All arguments except the data file list are optional.
         The data file list is required, unless the
         [-r afile] option is used, when the afile replaces 
         the data file list. If present,
         the "set" argument and the commands which follow it must
         come last.  All option flags must appear before the data file
         list, though the flags can come in any order.

    Here are two examples of command-line invocations of fourfit, with
    an explanation of what they do:

\texttt{fourfit -txas -m 1 -c control 018-234505 set mb\_win -0.0034 .004 freqs a b}

        Test mode, xwindow display, accounting switched on, cross
        power spectrum plot switched on, moderately verbose, use
        control file named "control" in current working directory,
        process all data in scan directory 018-234505, override
        multiband delay search window and select channels 'a' and
        'b' only.

\texttt{fourfit -r refr\_list -c control -d hardcopy -b AT:S}

        Process all data referenced by type 2 lines in the A-file
        named "refr\_list", use control file "control", print the
        fringe plot on the default printer, process only baseline
        AT frequency subgroup S.

\textbf{Option Flags:}

\begin{itemize}
\item [-a]
If specified, this option switches on accounting
of CPU time and wall-clock time used in the various
parts of fourfit.  When the program finishes, it
produces a summary of these timing statistics.
\item [-b BB:F]
Allows the user to override the control file
with a specification of the baseline and/or
frequency group to be processed.  The syntax is
flexible.  0, 1 or 2 characters before the colon
refer to the baseline (one character is interpreted
as a station), and 0 or 1 character after the colon
is interpreted as the frequency subgroup.  You can
use the control file wildcard character '?' in
the baseline, but remember to protect it from the
C-shell either by escaping it with a backslash '\\'
or enclosing the entire -b argument in single
quotes.  If you wish only to specify the baseline,
the colon may be omitted.  An error in the -b
flag argument causes the flag to be ignored, and
fourfit will continue execution.
\item[-c controlfile]
            Specifies the file which contains parameters
            to control the operation of the program.  If
            absent, fourfit will use only the file pointed to
            by the environment variable DEF\_CONTROL, which
            in turn defaults to \texttt{/dev/null} as defined
            in the HOPS/setup.csh file.  Any parameters
            set in a control file specified with the -c option
            override the default file values.  A description
            of the syntax of the control file, with an example,
            can be found later in this document.

\item[-d display\_device]
            Upon completion of a fringe fit, fourfit can
            optionally display the results using postscript.
            The valid choices for "display\_device" are:
                \begin{itemize}
                \item[]\textbf{diskfile:file.ps} save the plot in "file.ps"
                \item[]\textbf{hardcopy}         send the plot directly to lpr
                \item[]\textbf{pshardcopy}       print the plot via pplot\_print
                \item[]\textbf{xwindow}          show the plot in an X11 window
                \item[]\textbf{psscreen}         the same, but allow GS\_* options
                \end{itemize}

\item[-f first\_channel]
            overrides the default first channel (0) to facilitate
            plotting when there are more than 16 channels (see -n)

\item[-m value]
            This flag controls the verbosity of the program via
            the integer argument "value", which ranges from 3
            (virtually silent except for major errors) to -3 
            (incredibly verbose, of use only to the authors of 
            the program).  The default is 2.

\item[-n nchans]
            When there are many channels (see also the -f flag)
            this tells how many channels to put on the plot. Note
            that neither -f nor -n affect the actual fringe fit,
            just the plotting thereof.

\item[-p]
            Simplest way to pop up a fringe plot.
            This is equivalent to "-d psscreen".

\item[-r afile]
            Puts fourfit into "refringe" mode.  Fourfit refringing
            is driven by an A-file input, which overrides any 
            correlator root files directly specified on the command
            line (i.e. the latter are ignored when the -r flag
            is specified).  The input A-file, of which there can
            be only one, may contain root, corel or fringe lines,
            but only the fringe lines are used to determine which
            data to process, by baseline and frequency subgroup.
            Obviously, the -r flag is inconsistent with the -u
            flag, and specifying both is an error.  Note that for
            afiles using HP-1000 (version 1) line formats, fourfit
            has to pre-check the disk for the existence of the 
            type 2 files.  The data area is controlled by the
            DATADIR environment variable.  It defaults to the
            value of \$CORDATA.

\item[-s naps]
            This parameter controls how many AP's are merged
            together into each plotting segment. Thus the number
            of time points shown in the phase, amplitude, and
            validity plots is so controlled. Additionally, the
            ph/seg and amp/seg statistics are calculated based
            upon the stated number of AP's in each segment.

\item[-t]
            This flag places fourfit in test mode.  Everything
            works as normal, except that the output file is not
            written to disk, and the root file is not updated.
            This is useful when experimenting with different
            fringe-fitting strategies, in order to avoid cluttering
            up the disk.

\item[-u]
            Normally, fourfit processes all data consistent with
            the data file list and the control information.  When
            this flag is specified, fourfit will also check the
            information in the type-2100 record of the root to 
            see if the data have already been processed by fourfit.
            If so, the data in question are skipped.  The "u"
            stands for update mode.

\item[-x]
            This is equivalent to "-d xwindow".

\item[-P pq]
            Controls polarization processing. If pq is a 2 character
            string, then pq is one of sixteen cross-polarization 
            products, formed by p and q each being one
            of \{R, L, X, Y\}. Alternatively, pq can be a single
            character, 'I', which forms the Stokes pseudo-I mode
            combination of the products \{XX, YY, XY, YX\}.

\item[-T trefoffs]
            If this option is invoked, the fourfit reference
            time will be calculated by taking the nominal scan
            start time from the ovex file and adding trefoffs
            (which is an integer number of seconds) to it.

\item[-X]
            Forces fourfit to write cross-power spectra into
            type 230 records. This option is typically used for
            import into AIPS.
\end{itemize}

\textbf{Arguments:}

            data file list
            This mandatory argument or arguments tells fourfit
            which data files to process.  The format of the data
            file specification is the standard one for all MkIV
            software.  You may specify individual filenames, 
            scan directories which contain data files, 
            experiment directories, or any combination of
            these three.  In the latter two cases,
            fourfit will descend the directory tree looking for
            data files to add to its internal list of files to
            process.  Only root files need be specified.  The
            data files actually fringe-searched are determined
            by the combination of the root files specified and the
            restrictions imposed by the control file or control
            parameter list (see below).  In the absence of 
            such restrictions, all data associated with the 
            specified root files are processed.

            Beware of trying to specify too many files or scan
            directories, as it is possible to overflow the Unix
            argument list buffer on large experiments.  In such
            cases, specify the experiment directory instead.
