For some applications, such as mm-wave VLBI where the atmospheric coherence time scale is short, it is advantageous to be able to make a priori phase corrections in a fringe fit. Typically the phases will be derived through some other avenue, such as other instruments (e.g. water-vapor radiometers) or information from other polarization senses. For full flexibility, one can make corrections for each frequency channel as a function of time within each scan. 

The full time-frequency array is specified in a flat 2-dimensional ASCII file, one file per each station. Typically, though not necessarily, a single file would be used for an entire experiment. The file would be given a name that identifies the station whose data it contains. 

Each line in the file begins with an epoch, and then has entries for all channels for that epoch, represented by space-delimited phases in degrees. The UT epoch is in decimal days since the beginning of the year. (0300UT on Jan 2 is 1.125). Comments begin with an asterisk, and everything including and beyond an asterisk on each line is stripped prior to reading.

Thus an example of a rather short phase file (having only 4 channels, with data points each 10 sec) might appear as:

\texttt{
* Westford phases for chans at 221120.0 221152.0 221184.0 221216.0 \\
* \\
223.041667 13.2 -39.3 57.1 177.0 \\
223.041782 9.7 -36.2 77.0 -176.5 \\
223.041898 8.8 -37.7 92.4 -171.9 \\
}

The new fourfit control file commands to support this capability are: \\
\obeyspaces{\texttt{
adhoc\_phase file               * denotes that file mode pcal will be used \\
adhoc\_file <file\_name>          * relative or absolute file name containing phases \\
adhoc\_file\_chans <channel\_list>  * where channel\_list is ff chan codes \\
}}

An example of a block of a fourfit control file might be: \\
\texttt{
if station E \\
adhoc\_phase file \\
adhoc\_file pcal\_x3970\_wf \\
adhoc\_file\_chans abcd \\
}

{\bf Algorithm}

The time points are considered to be either instantaneous values, or the result of a linear average of the phase, symmetric about the specified epoch. The algorithm within fourfit performs a linear interpolation between the specified points, and then finds the average value of the piece-wise linear function so derived over each accumulation period in a fringe fit. If the ap data extends beyond the range of the piecewise linear pcal function, then the function is extrapolated linearly (if possible) to cover the ap.

The phase values are considered to be equivalent to measured pcal phase for a station. Thus the value will be removed from the data phase prior to the fringe fit. In other words, the data phase will be counter-rotated by the given pcal phase. The ad hoc file-based phases are applied in addition to any extracted phase cal (as invoked by pc\_mode multitone, normal, or manual). In order to zero out the
extracted pcal, one can simply use pc\_mode manual without any phases being specified in a pc\_phases command, as the default is 0.

Note that fourfit channel codes (abcde) have an implied context of the mk4 fileset it is being applied to. It is possible to recorrelate with a changed vex file, and have the resulting channel codes change. This shouldnÕt present a problem with a modicum of caution on the part of the fringe fitter.



