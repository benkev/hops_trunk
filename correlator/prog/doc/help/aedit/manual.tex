\documentstyle{article}
\title{}
\author{}
\addtolength{\textwidth}{1.5in}
\addtolength{\textheight}{1in}
\hoffset=-.8in
\voffset=-.5in
\setlength{\parindent}{0in}

\begin{document}
\pagestyle{empty}

\begin{center}
\Huge\bf
\underline{AEDIT version 4.1}
\vspace{2in}

\Large\rm
A program for editing, manipulating, visualizing and summarizing
MkIII/IV A-file format data
\vspace{2in}

by \\
Colin Lonsdale \\
Haystack Observatory \\
\vspace{.3in}

\em September 1994
\end{center}
\newpage

\begin{center}
\large\bf
\underline{New in 4.1}
\end{center}

Previous to this release of aedit, there were three types of A-file lines, 
corresponding to the three types of binary data files, root (type 0), 
corel (type 1) and fringe (type 2).  I have now added two new types of A-file 
lines.  They are type 3 (closure triangle) and type 4 (closure quad) lines, 
and they can be formed only by combining multiple type 2 (fringe) lines.  In 
this release of aedit I have implemented full support for type 3 lines only.  
Type 4 support will be added later at lower priority.  Closure triangle data 
can now be read, written, filtered, plotted, and interactively edited in much 
the same way as type 2 data.  You can get closure triangle data into aedit 
two ways.  First, you can read it in from a disk file using the standard read 
command.  Alternatively, you can read in some type 2, baseline data, and use 
the new close command to compute closure triangle data from the type 2 data.  
Either way, the closure data are fully plottable and manipulable.  It is 
important to note that aedit is (and should remain) the only HOPS program 
capable of generating closure data.  All closure triangle data in HOPS must 
ultimately originate from the close command in aedit operating on type 2 data, 
though it can be read, written and manipulated by aedit and other programs.


Because the data types are fundamentally different, type-3 closure triangle 
data and type-2 baseline data are stored internally in separate arrays.  They 
should also be stored on disk in separate files, so there is now a new twrite 
command (analogous to the existing write command), which writes out only closure 
triangle data.  If you really want to keep your baseline and closure data in one 
file, you can concatenate them and aedit will swallow the resulting file.  However, 
since other HOPS programs (such as average) will be expecting one or the other, 
and not both, it will probably be easier to keep them separate.

To accommodate closure triangle data, there are a couple of new filters you can 
set.  First, you can specify the closure triangles you want with the triangles 
command (station order is irrelevant), and second, you can specify a range of 
bispectral SNRs, using bsnrmin and bsnrmax.  The bispectral SNR calculated by 
the close command is the theoretical one based on the individual baseline SNRs.  
If the data come from a file instead, the bispectrum SNR may have been calculated 
differently (e.g. by program average).  Certain filters do not apply to certain data 
types, and are simply ignored.  For example, saying triangles ABC; edit inputs will 
have no effect on baseline (types 1 and 2) data.  All of the (relevant) filters 
familiar to you for baseline data, such as stations, quality code and so on work 
for closure triangle data.

There are some new options to the edit command which permit you to control the 
consistency of related baseline and triangle datasets.  The edit close triangles 
command removes triangle records which do not have all three legs of the triangle 
present in the baseline dataset.  The edit close baselines command, conversely, 
removes all baseline records which do not participate in at least one valid 
triangle record.  Just typing edit close does both operations, and ensures a 
fully consistent set of baselines and triangles.  The unflag command has been 
enhanced to accommodate the new types of flagging made possible by these filtering 
and editing changes.  Aedit makes no attempt to detect triangle redundancies 
(i.e. it will generate and/or accept N(N-1)(N-2)/6 triangles for N stations).  
While this may result in excessive amounts of closure triangle data for large 
experiments, it is necessary in order to preserve information in the low 
SNR case (where the triangles are not fully redundant), and besides simplifies 
what is already some quite convoluted logic for handling arbitrary input closure 
data files.

There have been a number of minor enhancements to the way various commands function.  
In all cases, these changes should cause no problems for existing aedit users.



\begin{center}
\large\bf
\underline{New in 4.0}
\end{center}

There have been many enhancements in ``aedit" since the release 3.0
in March 1992, some of which are major.  Below is a list of the most 
important enhancements.  The detailed documentation of the individual 
commands covers the numerous other, more minor changes.

\begin{enumerate}
\item Full support is now available for all three types of data record,
0, 1 and 2.  This applies to filtering, sorting, and other operations.
Two new ``edit" options are now available which remove data records
which lack parent or child data records (e.g. if a root record has no
associated corel or fringe records in the database, it can be edited
out with the command ``edit parents").  The ``family" sort option causes
data records to be grouped by root family when writing the data to disk
(normally, all type 2 records are written, then all type 1's, then all
type 0's).
\item Aedit now has significant experiment summary capabilities.  The
command ``ccread" causes a specified correlator control file to be read
into memory, after which one may invoke the ``psplot" command.  This is
a highly interactive experiment status browser and data editor, which
uses color-coded rectangles to represent data records.  You can tag or
edit data based on quality code, scantime or baseline, or by clicking on
individual data cells, and you can of course pop up fringe plots by 
clicking on any cell.  The display is intended to visually assist 
diagnosis of experiment-wide problems, among other things.
\item In order to circumvent the problem of limited numbers of parameters
in the standard A-file format, ``aedit" now supports parameter extraction
directly from the type-2 data files on disk.  Depending on circumstances,
you can now extract, plot, and write out any of a wide range of parameters.
You can also filter the data on the basis of the values of extracted parameters.
Phasecal information is now accessible only through the extracted
parameters.  The relevant commands are ``parameter", ``pwrite", ``plist"
and ``prange".
\item Plotting within ``aedit" has been improved significantly.  It is now
possible to plot any quantity, including extracted parameters if present,
against any other quantity (provided it makes rudimentary sense .. you
cannot plot closure rate versus baseline SNR, for example).  Specification
of the output device has been simplified for a subset of devices.  All
plots remain fully active, in the sense that on appropriate devices you can
edit data with the interactive cursor, or pop up fringe plots on Xwindow
devices.  New commands of relevance include ``xscale", ``yscale", ``reference"
and ``remote".
\item Along with all other MkIV programs, ``aedit" supports the new version 2
A-file format.  This format features a somewhat more useful mix of quantities,
with additional precision in many cases, and is to be preferred where
possible.  ``Aedit" can write the data out in any A-file format (see the
command ``outversion").
\end{enumerate}

\begin{center}
\Large\bf
\underline{GENERAL DESCRIPTION OF AEDIT}
\end{center}

The program ``aedit" is a general purpose A-file manipulation
program.  The information present in one or more A-files may
be plotted, filtered, sorted and edited in a variety of ways, 
before being written out in the form of a new A-file.  The user
interface to the program is presently implemented only for
ASCII terminals, but many functions of the program interact
with the user via a graphical interface.

Commands are given to aedit by keyboard, and a full minimum
match capability is supported for all aedit names.  Multiple
commands are allowed on one input line, the only requirement
being that commands are separated by the semicolon ``;" 
character.  Commands typically consist of the command name
followed by 0 or more arguments.  The arguments are separated
from the command name and each other by either spaces or
commas.  Aedit can handle long lines, but it is of course bad
practice to wrap lines on terminals in general.  Upwards of
about 250 characters may start to cause problems even for aedit.
In general, aedit is not case sensitive.  Case sensitivity
is needed for UNIX filenames and for certain quantities from 
A-files (station codes, frequency codes, source names).

Aedit uses the concept of inputs.  That is, you set up
certain variables in the program that determine how the ``action"
commands will behave.  Most of the commands that aedit
understands are of the input-setting variety.  Many are quite
particular about syntax, and will complain if the user types
nonsense (e.g. timerange).  At any time, the current state of
the input parameters can be listed on the screen with the
command ``inputs".

When aedit reads data from an A-file, it parses the ascii
information and stores it as binary data in memory.  This allows
very rapid manipulation of the data once read in, with
seemingly complex tasks appearing to be instantaneous.  There
is a flag field associated with each A-file line in memory,
and these flags are manipulated by the edit and unflag commands.
A full description of the data currently in memory can be
obtained with the ``summary" command.  This is essential when
deciding on plotting and editing options.  The ``write" command
ignores flagged data, permitting aedit to be used as a simple
and efficient filtering program.

A command ``run" is available, which provides a flexible and 
general command file capability.  Nesting of command files to a 
depth of 10 is allowed.  The ``run" command executes in batch mode, 
and cursor operations are therefore disabled.

Aedit features a shell escape.  By starting an input line with the
character ``!", you can access standard UNIX commands outside of
aedit.  You can escape to a complete new shell by typing ``!csh" or
``!sh", and when you have finished, return to aedit where you left
off by typing cntl-D.  This feature is useful for spooling plot
files to a printer, preparing run files, running ``alist" to prepare
new data for aedit, and any other tasks that you wish to perform
without terminating the aedit session.

Plotting is implemented by using the PGPLOT package from CalTech.
The output device may be specified with the ``device" command, or
you may leave ``device" at the default value (``?"), which will
cause PGPLOT to query you for a device at the time of plotting.
Your response will then be automatically entered into the ``device"
input.  A list of available device types can be obtained by responding
with a query.  For more information of devices, see ``help device"

Aedit comes with full on-line help.  In general, the syntax is
``help command", but just ``help" will work.

The command line for aedit is ``aedit [-x] [-r filename] [-f filename]",
where ``-x",``-r", and ``-f" are optional.  The ``-x" option means start
up the xwindow interface (not yet supported).  The ``-r" option means
execute the specified run file on startup, and must be immediately
followed by the name of a file containing valid aedit commands.  The ``-f" 
option means ``read this(ese) data file(s) on startup", and must be 
immediately followed by a standard, wildcardable UNIX filename specifier 
or specifiers.  In this way, you can read many files at once into aedit
without going through a laborious one-at-a-time ``read" cycle within
the program. If specified, the ``-f" flag must be the last flag.

Below is a list of all current aedit commands:

\begin{verbatim}
Action commands:
----------------
batch        clear       close       edit         exit
fplot        help        inputs      nobatch      parameter
plist        plot        pwrite      read         run
setyear      sort        summary     twrite       unflag
unsort       write       zoom


Plot control commands:
----------------------
grid         xscale      yscale      axis         mode
reference    remote


Data selection commands:
------------------------
baselines    bsnrmin     bsnrmax     experiment  fraction
frequencies  length      nfreq       prange      procrange
qcodes       snrmax      snrmin      sources     stations
timerange    triangles   type


Experiment overview commands/parameters
---------------------------------------
schedread    psplot      psfile


IO control commands:
--------------------
device       outversion

\end{verbatim}

For further information, see the individual help files for the
above commands.

\newpage

\begin{center}
\Large\bf
\underline{COMMAND DESCRIPTIONS}
\end{center}
\vspace{.3in}

\begin{tabbing}
Titlexxxxxxxxxxxxxxx \= \kill
\underline{COMMAND} \> {\bf    help} \\
\end{tabbing}

\begin{tabbing}
Titlexxxxxxxxxxxxxxx \= \kill
\underline{TYPE} \> {\bf            Action} \\
\end{tabbing}

\begin{tabbing}
Titlexxxxxxxxxxxxxxx \= \kill
\underline{SYNTAX} \> {\bf          ``help `subject'"} \\
\end{tabbing}

\underline{DESCRIPTION}
\begin{list}{}{\setlength{\leftmargin}{0.5in}
     \setlength{\rightmargin}{0in}}
\item
Writes the help file pertaining to ``subject" on the 
terminal, under pagination control.  ``Subject" is
presently any command name, plus ``general".
\item
Action commands:
\item
\begin{verbatim}
        Name    argument(s)     Description
        ----    -----------     -----------
        clear   data            Erase all data from memory
                close		Erase all triangle data from memory
                inputs          Reset input settings to default
                plot            Clear screen or eject page
                all             All three clear functions
        close                   Generate triangle data from type-2 data
        edit    cursor          Zap points on screen with cursor
                inputs          Remove points that don't fit inputs
                duplicates      Remove duplicate points with various priorities
                parents         Remove childless parent records
                children        Remove orphan child records
                close triangles Remove triangle record without baselines
                close baselines Remove baseline record without triangles
                close           Perform both of the above commands
        exit                    End aedit session
        fplot                   Pop up Xwindow fringe plots
        inputs  plot/filter     Print current input settings on screen
        parameter number(s)     Extract parameters from type-2 files on disk
        plist                   List extracted parameters in memory
        plot                    Plot current data according to inputs
        pwrite  filename        Write extracted parameters to filename
        read    filename        Read in data from filename
        run     filename        Execute commands in filename
        setyear number          Manually reset year of scan throughout data
        sort    key             Sort data according to various keys
        summary                 Display a summary of all unflagged data
        twrite  filename        Write (edited, sorted) type 3 data to filename
        unflag  string          Removes flags applied for various reasons
        unsort                  Restore original sort order (as read in)
        write   filename        Write (edited, sorted) type 0,1,2 data to filename
        zoom                    Display details of cursor-selected points


Plot control parameters:

        Name    argument(s)     Description
        ----    -----------     -----------
        axis    string          Set variable to plot on Y axis
        grid    a,b             Divide screen/page a times b subplots
        xscale  min,max         Set X-axis scale
        yscale  min,max         Set Y-axis scale
        mode    split/nosplit   Do/don't do 1 plot per source
        reference               Use reference antenna in baseline plots
        remote                  Use remote antenna in baseline plots


Data selection parameters:

        Name            argument(s)     Description
        ----            -----------     -----------
        baselines       AB,BC,AC ....   Use only these type 2 baselines
        triangles       ABC,DEF,ADE ..  Use only these type 3 triangles
        experiment      expt #          Use only data from this experiment
        frequencies     S,X,K, ....     Use only data at these frequencies
        fraction        nn% ....        Use only scans with >nn% good data
        length          number          Use only scans > number secs or blocks
        nfreq           <>= nn          Use only scans with <>= nn frequencies
        qcodes          5-9,D ...       Use only data with these quality codes
        snrmax          number          Use only type 2 data with snr < number
        snrmin          number          Use only type 2 data with snr > number
        bsnrmax         number          Use only type 3 data with bsnr < number
        bsnrmin         number          Use only type 3 data with bsnr > number
        sources         name1,name2...  Use only data on these sources
        stations        A,B,C,D ....    Use only data from these stations
        timerange       yyddd-hhmmss, yyddd-hhmmss      
                                        Use only data in time range
        prange          n, min, max     Use only data with parameter n in range
        procrange       yyddd-hhmm, yyddd-hhmm  
                                        Use only data in procdate range
        type            0 1 2 3         Use only data of these types


Experiment overview commands/parameters:

        Name            argument(s)     Description
        ----            -----------     -----------
        ccread          filename        Read specified CC file into memory
        psplot                          Display active plot of experiment
        psfile          filename        Write experiment summary to filename

I/O control parameters:

        Name            argument(s)     Description
        ----            -----------     -----------
        device          string          Plotting device for PGPLOT
        outversion      number          Determines output A-file format version


Miscellaneous:

        Name    argument(s)     Description
        ----    -----------     -----------
        batch   none            Disables interactive confirmation querys
        nobatch none            Enables interactive confirmation querys
\end{verbatim}
\end{list}
\vspace{.2in}

\begin{tabbing}
Titlexxxxxxxxxxxxxxx \= \kill
\underline{COMMAND} \> {\bf    axis} \\
\end{tabbing}

\begin{tabbing}
Titlexxxxxxxxxxxxxxx \= \kill
\underline{TYPE} \> {\bf            Plot control} \\
\end{tabbing}

\begin{tabbing}
Titlexxxxxxxxxxxxxxx \= \kill
\underline{SYNTAX} \> {\bf          ``axis y-axis $<$x-axis$>$"} \\
\end{tabbing}

\underline{DESCRIPTION}
\begin{list}{}{\setlength{\leftmargin}{0.5in}
     \setlength{\rightmargin}{0in}}
\item
This command specifies what the X and Y axes of the next ``plot" command
will be.  The y-axis specifier is mandatory, the x-axis one optional.  If
the x-axis specifier is omitted, it is assumed to be ``scantime".  After
the ``axis" command, the ``plot" command with no arguments causes the specified
axes to be plotted on the current output device (see ``help device").  If
the ``plot" command is issued with arguments, those arguments override the
axis settings made by the ``axis" command, and reset the plot inputs.  The
arguments to the ``plot" command are identical to those described below.
Axis specifiers are case-insensitive.
\item
Valid axis specifiers, with notes, are listed below:
\item
\begin{verbatim}
Minmatch string         Quantity                        Notes
---------------         --------                        -----
scantime (or time)      data time tag                   default X-axis
pcal_phase(n)           Ref/remote pcal phase           Must extract first
                                                        (see help param)
                                                        n is integer array
                                                        element index.
pcal_diff(n)            Ref/remote pcal phase diff      Must extract first
                                                        Relative to channel 1
pcal_amp(n)             Ref/remote pcal amplitude       Must extract first
                                                        Sign encodes pcal mode
error_rate              Ref/remote error rate           Must extract first
elevation               Elevation of telescope          If baseline plot, 
                                                        ref/remote controlled
                                                        by "reference", "remote"
                                                        commands
azimuth                 Azimuth of telescope            Same as elevation
snr                     SNR
amplitude               Correlation amplitude
phase                   Residual scan phase
sbdelay                 Singleband delay
mbdelay                 Multiband delay                 Ambiguities removed
drate                   Delay rate
cphase                  Closure phase                   Uses totals
crate                   Closure rate
csbdelay                Closure singleband delay
cmbdelay                Closure multiband delay         Ambiguities removed
campl                   Closure amplitude               NYI
u                       U in megalambda 
v                       V in megalambda 
uvdist                  UV distance in megalambda
param?                  extracted parameter value       The '?' is an integer
                                                        which specifies which
                                                        parameter to use.  To
                                                        get a list, use "plist".
\end{verbatim}
\end{list}
\vspace{.2in}

\begin{tabbing}
Titlexxxxxxxxxxxxxxx \= \kill
\underline{COMMAND} \> {\bf 	baselines} \\
\end{tabbing}

\begin{tabbing}
Titlexxxxxxxxxxxxxxx \= \kill
\underline{TYPE} \> {\bf 		Data selection} \\
\end{tabbing}

\begin{tabbing}
Titlexxxxxxxxxxxxxxx \= \kill
\underline{SYNTAX} \> {\bf 		``baselines AB BC CD AC"} \\
\end{tabbing}

\underline{DESCRIPTION}
\begin{list}{}{\setlength{\leftmargin}{0.5in}
     \setlength{\rightmargin}{0in}}
\item
Sets the baseline data selection parameter in the inputs.  Only
those baselines specified will pass the filter-applying operations
of edit inputs, read, and plot.  Typing ``baselines" without
arguments removes any restrictions on allowed baselines.
\end{list}
\vspace{.2in}

\begin{tabbing}
Titlexxxxxxxxxxxxxxx \= \kill
\underline{COMMAND} \> {\bf 	batch} \\
\end{tabbing}

\begin{tabbing}
Titlexxxxxxxxxxxxxxx \= \kill
\underline{TYPE} \> {\bf 		Miscellaneous} \\
\end{tabbing}

\begin{tabbing}
Titlexxxxxxxxxxxxxxx \= \kill
\underline{SYNTAX} \> {\bf 		``batch"} \\
\end{tabbing}

\underline{DESCRIPTION}
\begin{list}{}{\setlength{\leftmargin}{0.5in}
     \setlength{\rightmargin}{0in}}
\item
Disables confirmation mechanism, for running in batch mode.
If you intend to plot data in batch mode, remember to set
the plot device in your runfile, as the default PGPLOT query
mechanism is disabled in this mode.
\end{list}
\vspace{.2in}

\begin{tabbing}
Titlexxxxxxxxxxxxxxx \= \kill
\underline{COMMAND} \> {\bf 	snrmax} \\
\end{tabbing}

\begin{tabbing}
Titlexxxxxxxxxxxxxxx \= \kill
\underline{TYPE} \> {\bf 		Data selection} \\
\end{tabbing}

\begin{tabbing}
Titlexxxxxxxxxxxxxxx \= \kill
\underline{SYNTAX} \> {\bf 		``snrmax 20"} \\
\end{tabbing}

\underline{DESCRIPTION}
\begin{list}{}{\setlength{\leftmargin}{0.5in}
     \setlength{\rightmargin}{0in}}
\item
Sets the maximum snr which will pass the filters applied
in edit inputs, read, and plot.  Typing ``snrmax" without
arguments removes any upper bound on snr.
\end{list}
\vspace{.2in}

\begin{tabbing}
Titlexxxxxxxxxxxxxxx \= \kill
\underline{COMMAND} \> {\bf 	snrmin} \\
\end{tabbing}

\begin{tabbing}
Titlexxxxxxxxxxxxxxx \= \kill
\underline{TYPE} \> {\bf 		Data selection} \\
\end{tabbing}

\begin{tabbing}
Titlexxxxxxxxxxxxxxx \= \kill
\underline{SYNTAX} \> {\bf 		``snrmin 20"} \\
\end{tabbing}

\underline{DESCRIPTION}
\begin{list}{}{\setlength{\leftmargin}{0.5in}
     \setlength{\rightmargin}{0in}}
\item
Sets the minimum snr which will pass the filters applied
in edit inputs, read, and plot.  Typing ``snrmin" without
arguments removes any lower bound on snr.
\end{list}
\vspace{.2in}

\begin{tabbing}
Titlexxxxxxxxxxxxxxx \= \kill
\underline{COMMAND} \> {\bf 	ccread} \\
\end{tabbing}

\begin{tabbing}
Titlexxxxxxxxxxxxxxx \= \kill
\underline{TYPE} \> {\bf 		Experiment overview} \\
\end{tabbing}

\begin{tabbing}
Titlexxxxxxxxxxxxxxx \= \kill
\underline{SYNTAX} \> {\bf 		``ccread filename"} \\
\end{tabbing}

\underline{DESCRIPTION}
\begin{list}{}{\setlength{\leftmargin}{0.5in}
     \setlength{\rightmargin}{0in}}
\item
This command reads a correlator control file into memory, thus
allowing aedit to compare what ``should" be present to what is
actually present.  Execution of this command is a prerequisite
for ``psplot" and ``psfile".
\end{list}
\vspace{.2in}

\begin{tabbing}
Titlexxxxxxxxxxxxxxx \= \kill
\underline{COMMAND} \> {\bf 	clear} \\
\end{tabbing}

\begin{tabbing}
Titlexxxxxxxxxxxxxxx \= \kill
\underline{TYPE} \> {\bf 		Action} \\
\end{tabbing}

\begin{tabbing}
Titlexxxxxxxxxxxxxxx \= \kill
\underline{SYNTAX} \> {\bf 		``clear data"} \\
\> {\bf 		``clear close"} \\
\> {\bf 		``clear inputs"} \\
\> {\bf 		``clear plot"} \\
\> {\bf 		``clear all"} \\
\end{tabbing}

\underline{DESCRIPTION}
\begin{list}{}{\setlength{\leftmargin}{0.5in}
     \setlength{\rightmargin}{0in}}
\item
``Clear data" removes all the data from memory, and returns
array space to the system.  Since any active plot no longer
refers to data in memory after this operation, the plot is
rendered inactive.
\item
``Clear close" is analogous to ``clear data", but removes only
closure data.
\item
``Clear inputs" changes all the values listed by the ``inputs"
command to their default values.  Typically, this means
data selection parameters are set to pass all data, and plots
revert to self-scaling.
\item
``Clear plot" flushes the current plot, and renders a plot on
an interactive device inactive.
\item
``Clear all" simultaneously performs all the above operations.
\end{list}
\vspace{.2in}

\begin{tabbing}
Titlexxxxxxxxxxxxxxx \= \kill
\underline{COMMAND} \> {\bf 	close} \\
\end{tabbing}

\begin{tabbing}
Titlexxxxxxxxxxxxxxx \= \kill
\underline{TYPE} \> {\bf 		Action} \\
\end{tabbing}

\begin{tabbing}
Titlexxxxxxxxxxxxxxx \= \kill
\underline{SYNTAX} \> {\bf 		close} \\
\end{tabbing}

\underline{DESCRIPTION}
\begin{list}{}{\setlength{\leftmargin}{0.5in}
     \setlength{\rightmargin}{0in}}
\item
Causes all type 2 baseline data in memory to be examined, and 
closure triangles to be formed.  The result is a set of type-3 
records in memory, which can then be plotted, edited, filtered 
and written out to disk just like other data.  The close command 
will refuse to generate closure data unless there are currently 
no triangle records present.
\end{list}
\vspace{.2in}

\begin{tabbing}
Titlexxxxxxxxxxxxxxx \= \kill
\underline{COMMAND} \> {\bf 	device} \\
\end{tabbing}

\begin{tabbing}
Titlexxxxxxxxxxxxxxx \= \kill
\underline{TYPE} \> {\bf 		IO control} \\
\end{tabbing}

\begin{tabbing}
Titlexxxxxxxxxxxxxxx \= \kill
\underline{SYNTAX} \> {\bf 		``device postscript"} \\
\> {\bf 		``device hpgl"} \\
\> {\bf 		``device xwindow"} \\
\> {\bf 		``device name/device"} \\
\end{tabbing}

\underline{DESCRIPTION}
\begin{list}{}{\setlength{\leftmargin}{0.5in}
     \setlength{\rightmargin}{0in}}
\item
Sets the device type used for plotting.  The available devices
are those accessible to the PGPLOT library, which as of February
1994 consisted of tektronix emulators, an Xwindow graphics window,
Hewlett-Packard Laserjet printers, and postscript printers.
\item
The ``device" command can be invoked in 2 ways.  First, there
are 4 keywords that are recognized, namely ``ppostscript", 
``lpostscript", ``hpgl" and ``xwindow".  If one of these keywords is 
specified, aedit will use the corresponding device in a transparent,
automatic way.  The hardcopy options, ``ppostscript", ``lpostscript"
and ``hpgl", send the plot output to whatever printer is specified
by the shell script ``aedit\_plot".  The ``hpgl" output is portrait,
while the postscript output can be either landscape (lpostscript)
or portrait (ppostscript).  The printing is done immediately, 
without the need for a ``clear plot" command or separate invocation 
of a printer job.
\item
The second method involves direct access to the PGPLOT device
specification mechanism, as described in detail below.
\item
The construction of the argument is in two parts.  The first
part is the specific name of the output file or device,  The
second part specifies the type of device.  The former can
be a standard UNIX filename, such as ``plot01.3C345", but 
subdirectory specifiers (i.e. filenames with ``/" in them) are
special because PGPLOT is looking for a ``/" to separate
the two parts of the device specifier.  You must ``hide" the
UNIX ``/" characters from PGPLOT by enclosing the filename in
double quotes, so that a valid specification for a workstation
tektronix emulator might be ```/dev/ttyp2"/te'.
\item
The default filename for interactive devices is the users
terminal, whilst for the hardcopy devices, it is ``PGPLOT.device".
PGPLOT translates filenames to upper case on output.
\item
The second part, the device type, follows a ``/", and a complete
list of possibilities can be viewed by setting the device equal
to ``?", the default setting.  The names are minimum matchable 
(e.g. ``/te" will work).  
\end{list}
\vspace{.2in}

\begin{tabbing}
Titlexxxxxxxxxxxxxxx \= \kill
\underline{COMMAND} \> {\bf 	edit} \\
\end{tabbing}

\begin{tabbing}
Titlexxxxxxxxxxxxxxx \= \kill
\underline{TYPE} \> {\bf 		Action} \\
\end{tabbing}

\begin{tabbing}
Titlexxxxxxxxxxxxxxx \= \kill
\underline{SYNTAX} \> {\bf 		``edit inputs unflagged"} \\
\> {\bf 		``edit inputs all"} \\
\> {\bf 		``edit inputs"} \\
\> {\bf 		``edit cursor"} \\
\> {\bf 		``edit duplicates procdate"} \\
\> {\bf 		``edit duplicates qcode"} \\
\> {\bf 		``edit duplicates snr"} \\
\> {\bf 		``edit parents"} \\
\> {\bf 		``edit children"} \\
\> {\bf 		``edit close baselines"} \\
\> {\bf 		``edit close triangles"} \\
\> {\bf 		``edit close"} \\
\end{tabbing}

\underline{DESCRIPTION}
\begin{list}{}{\setlength{\leftmargin}{0.5in}
     \setlength{\rightmargin}{0in}}
\item
Sets flags in the data records according to a variety of
circumstances.  These flags can be selectively unset with
the ``unflag" command.
\item
``Edit inputs" sets a flag bit in each data record for each
data selection input parameter which excludes that data point.
Thus, a scan may pass the input filter for stations, but fail
that for baselines.  The baseline bit would be set, but the
station bit would not.  Any set bit in the flag field causes
the scan to be flagged (i.e. it will not be plotted or written
to an output file).  The ``unflagged" qualifier applies the filter
only to currently unflagged data.  The default ``all" qualifier
sets flag bits if appropriate even in currently flagged data.
\item
``Edit cursor" enables the cursor on an interactive graphics device
upon which data has been displayed using ``plot".  The user may
type any character (except `x', `X', `a', `A', `b' or `B' ..  see 
below) on the keyboard to edit out the point nearest the cursor.
The cursor must be inside the border of a plot, and must be twice 
as close to the target point than any other point for success.  
Failure to meet these conditions results in an appropriate error 
message.    
\item
Alternatively, the user may define an area on the plot within which
all points are to be edited out.  This is accomplished by typing
`a' or `A' to locate the bottom left corner of a rectangle, and
`b' or `B' to locate the top right corner.  Unfortunately, as yet
there is no visual indication of the current location of the 
rectangle.  This may be changed in future releases.
\item
On devices which are not capable of erasing points from the screen
(e.g. tektronix emulators), the edited points are marked by being
overwritten by a solid square.
\item
Do not use the mouse buttons on workstation tektronix emulators - 
these return multiple characters which may confuse the program.
The cursor editing mode is terminated by typing the character `x' 
or `X' on the keyboard.  The same may also be true of Xwindow
screens.
\item
``Edit duplicates" removes duplicate scans from the database, 
ignoring flagged scans.  The term ``duplicate" refers to identical
baseline, scan time, frequency code, experiment number and source.
The second argument determines which scan aedit will retain.  If
``procdate" is specified, it will keep the most recent processing.
If ``qcode" is specified, the ``best" quality code scan is kept. If
``snr" is specified, the highest snr scan is kept.
\item
WARNING: Since ``edit duplicates" ignores flagged scans, unflagging
data may generate more duplicates.  Similarly, reading in more
data may do the same.  In such circumstances, the recommended
course is to ``unflag duplicates" and rerun ``edit duplicates".
\item
``Edit duplicates" and ``edit cursor" operate only on type 2 data.
\item
``Edit parents" and ``edit children" allow you to construct a consistent
set of type-0, type-1 and type-2 data, such as may be needed for data
export, archiving, and so on.  ``Edit parents" flags all type-0 and
type-1 data records which have no corresponding children (i.e. types
1 or 2 for root records, and type 2 for corel records).  ``Edit children"
removes all ``orphan" type 1 and 2 records (i.e. those type-2 records
with neither parent root nor corel records, and type-1 records without
parent root records).
\item
``Edit close" flags type 2 baseline and type 3 triangle records according
to whether or not the two types of data are consistent with each other.
The baseline form of the command flags all baseline records which do
not appear in any unflagged triangle records.  The triangle form of the
command flags all triangle records for which all three constituent
baseline records are not present and unflagged.  Applying both forms
(as happens if the second argument is omitted) results in a fully
consistent set of baseline and triangle records in memory.
\end{list}
\vspace{.2in}

\begin{tabbing}
Titlexxxxxxxxxxxxxxx \= \kill
\underline{COMMAND} \> {\bf 	exit} \\
\end{tabbing}

\begin{tabbing}
Titlexxxxxxxxxxxxxxx \= \kill
\underline{TYPE} \> {\bf 		Action} \\
\end{tabbing}

\begin{tabbing}
Titlexxxxxxxxxxxxxxx \= \kill
\underline{SYNTAX} \> {\bf 		``exit" (no arguments)} \\
\end{tabbing}

\underline{DESCRIPTION}
\begin{list}{}{\setlength{\leftmargin}{0.5in}
     \setlength{\rightmargin}{0in}}
\item
Terminates the current aedit session.  All data currently in memory
is lost.  The plot device, if open, is closed and the plot flushed.
\end{list}
\vspace{.2in}

\begin{tabbing}
Titlexxxxxxxxxxxxxxx \= \kill
\underline{COMMAND} \> {\bf 	experiment} \\
\end{tabbing}

\begin{tabbing}
Titlexxxxxxxxxxxxxxx \= \kill
\underline{TYPE} \> {\bf 		Data selection} \\
\end{tabbing}

\begin{tabbing}
Titlexxxxxxxxxxxxxxx \= \kill
\underline{SYNTAX} \> {\bf 		``experiment 1953"} \\
\end{tabbing}

\underline{DESCRIPTION}
\begin{list}{}{\setlength{\leftmargin}{0.5in}
     \setlength{\rightmargin}{0in}}
\item
Sets the experiment input data selection parameter.  Only one
experiment number may be specified at one time.  Scans which do
not belong to the specified experiment number will not pass the
filters applied by edit inputs, read, and plot.  Typing ``experiment"
without arguments removes any restriction on experiment number.
\end{list}
\vspace{.2in}

\begin{tabbing}
Titlexxxxxxxxxxxxxxx \= \kill
\underline{COMMAND} \> {\bf 	fplot} \\
\end{tabbing}

\begin{tabbing}
Titlexxxxxxxxxxxxxxx \= \kill
\underline{TYPE} \> {\bf 		Action} \\
\end{tabbing}

\begin{tabbing}
Titlexxxxxxxxxxxxxxx \= \kill
\underline{SYNTAX} \> {\bf 		``fplot" (no arguments)} \\
\end{tabbing}

\underline{DESCRIPTION}
\begin{list}{}{\setlength{\leftmargin}{0.5in}
     \setlength{\rightmargin}{0in}}
\item
Enables the cursor on an active plot on an interactive
graphics device.  The user selects a point by positioning
the cursor and typing any character except `x' or `X', and 
the program pops up a fringe plot on the screen.  This fringe
plot can be dismissed with the `q' key, and the cursor is then
ready for the next point.
\item
Note that the ``fplot" command works only in an X-windows
environment.  Also, to display a fringe plot, aedit must be
able to locate the type-2 (fringe) file on disk from which the
A-file data were generated.  By default, it looks in the
CORDATA area, but if the DATADIR environment variable is set,
it looks there instead.
\item
``Fplot" is terminated by typing an `x' or `X'.
\end{list}
\vspace{.2in}

\begin{tabbing}
Titlexxxxxxxxxxxxxxx \= \kill
\underline{COMMAND} \> {\bf 	fraction} \\
\end{tabbing}

\begin{tabbing}
Titlexxxxxxxxxxxxxxx \= \kill
\underline{TYPE} \> {\bf 		Data selection} \\
\end{tabbing}

\begin{tabbing}
Titlexxxxxxxxxxxxxxx \= \kill
\underline{SYNTAX} \> {\bf 	  e.g.	``fraction $>$ 8"} \\
\> {\bf 	    or	``fraction $<$= 60\%"} \\
\end{tabbing}

\underline{DESCRIPTION}
\begin{list}{}{\setlength{\leftmargin}{0.5in}
     \setlength{\rightmargin}{0in}}
\item
Sets the fraction of the data processed for this scan which
will pass the filtering functions applied in read, edit, and
plot.  The syntax is quite forgiving.  The requirements are
that there be an inequality operator, possibly followed by
an equals sign, followed by a sensible number, possibly followed
by a percent sign.  If the percent sign is missing, the number
is interpreted as tenths of the scheduled data, instead of a 
percentage.  Spaces are irrelevant.
\item
If ``fraction" is typed with no arguments, or with
just ``0" as an argument, all restrictions on the fraction of
data processed are removed.
\item
Note that this filter option operates on the value of the last
digit in the ESDESP field of the A-file format, which is placed
there by FRNGE or fourfit.  Before the implementation of baseline-
dependent scan lengths in the schedule files, this number was 
unreliable.  Also, being only a single digit, this quantity is 
only accurate to the nearest 10\%, so more precise values entered 
with the fraction command are rounded off.
\item
\end{list}
\vspace{.2in}

\begin{tabbing}
Titlexxxxxxxxxxxxxxx \= \kill
\underline{COMMAND} \> {\bf 	frequencies} \\
\end{tabbing}

\begin{tabbing}
Titlexxxxxxxxxxxxxxx \= \kill
\underline{TYPE} \> {\bf 		Data selection} \\
\end{tabbing}

\begin{tabbing}
Titlexxxxxxxxxxxxxxx \= \kill
\underline{SYNTAX} \> {\bf 		``frequencies XS, C"} \\
\end{tabbing}

\underline{DESCRIPTION}
\begin{list}{}{\setlength{\leftmargin}{0.5in}
     \setlength{\rightmargin}{0in}}
\item
Enters a list of allowed frequencies into the inputs.  All alphabetic
characters are accepted, in any order, lower or upper case, with or
without spaces or commas.  Duplicate characters are ignored.  Scans
which involve frequencies not in this list will fail the filter tests
applied by edit inputs, read, and plot.  Typing ``frequencies" without
arguments removes any limitations on frequencies.
\end{list}
\vspace{.2in}

\begin{tabbing}
Titlexxxxxxxxxxxxxxx \= \kill
\underline{COMMAND} \> {\bf 	grid} \\
\end{tabbing}

\begin{tabbing}
Titlexxxxxxxxxxxxxxx \= \kill
\underline{TYPE} \> {\bf 		Plot control} \\
\end{tabbing}

\begin{tabbing}
Titlexxxxxxxxxxxxxxx \= \kill
\underline{SYNTAX} \> {\bf 		``grid n1 n2" (n1, n2 are integers - n1 $<$= 2, n2 $<$= 10)} \\
\end{tabbing}

\underline{DESCRIPTION}
\begin{list}{}{\setlength{\leftmargin}{0.5in}
     \setlength{\rightmargin}{0in}}
\item
This sets the parameter which determines how many subplots
appear horizontally and vertically on the plotting surface.
The default is one in each direction, the maximum is 2
horizontally and 10 vertically.  The character size scales
with the number of vertical plots to keep things readable.
\end{list}
\vspace{.2in}

\begin{tabbing}
Titlexxxxxxxxxxxxxxx \= \kill
\underline{COMMAND} \> {\bf    inputs} \\
\end{tabbing}

\begin{tabbing}
Titlexxxxxxxxxxxxxxx \= \kill
\underline{TYPE} \> {\bf            Action} \\
\end{tabbing}

\begin{tabbing}
Titlexxxxxxxxxxxxxxx \= \kill
\underline{SYNTAX} \> {\bf          ``inputs plot"} \\
\> {\bf             or  ``inputs filter"} \\
\> {\bf         or just ``inputs"} \\
\end{tabbing}

\underline{DESCRIPTION}
\begin{list}{}{\setlength{\leftmargin}{0.5in}
     \setlength{\rightmargin}{0in}}
\item
For use in ascii-terminal interface mode only.  Places a
summary of the current aedit input settings on the screen.
The plot and filter options result in a display of only 
those inputs pertaining to plotting and data filtering
respectively, while the default produces a display of all
input parameters.
\item
An example is shown below.
\item
\begin{verbatim}
        ****************
        | AEDIT INPUTS |
        ****************


DATA FILTER PARAMETERS
----------------------
Timerange:   88124-125700 to  88126-071500
Procrange:   89119-0223 to  89119-1256
Stations:    ABNT
Baselines:   AB BN TN 
Triangles:   No restriction specified
Frequencies: XS
Experiment:  1996
Qcodes:      56789AD
Type:        2
Snrmin:      10
Snrmax:      40
Bis_snrmin:  None specified
Bis_snrmax:  None specified
Sources:     3C345 3C273 OJ287
Length:      30 
Fraction:    No restriction specified
Nfreq:       No restriction specified
Outversion:  0 (i.e. same as that read in)


PLOTTING PARAMETERS
-------------------
Axis:        Plot amplitude against scan_time
              (station-based quantities use reference antenna)
Grid:        plot with 2 horizontal and 5 vertical subplots
Y-scale:     Plot between extrema of data
X-scale:     Plot between extrema of data
Scale:       Plot between mbdelay = -20.000000 and 20.000000
Mode:        Nosplit (multiple sources per plot)
Device:      Device for graphics output = xwindow
\end{verbatim}
\end{list}
\vspace{.2in}

\begin{tabbing}
Titlexxxxxxxxxxxxxxx \= \kill
\underline{COMMAND} \> {\bf 	length} \\
\end{tabbing}

\begin{tabbing}
Titlexxxxxxxxxxxxxxx \= \kill
\underline{TYPE} \> {\bf 		Data selection} \\
\end{tabbing}

\begin{tabbing}
Titlexxxxxxxxxxxxxxx \= \kill
\underline{SYNTAX} \> {\bf 		``length 20"} \\
\end{tabbing}

\underline{DESCRIPTION}
\begin{list}{}{\setlength{\leftmargin}{0.5in}
     \setlength{\rightmargin}{0in}}
\item
Sets the minimum scan length in seconds
which will pass the filters applied in edit inputs, 
read, and plot.  Typing ``length" without arguments 
removes any limitation on scan length.  This data
selection parameter applies only to type-2 data.
\end{list}
\vspace{.2in}

\begin{tabbing}
Titlexxxxxxxxxxxxxxx \= \kill
\underline{COMMAND} \> {\bf 	mode} \\
\end{tabbing}

\begin{tabbing}
Titlexxxxxxxxxxxxxxx \= \kill
\underline{TYPE} \> {\bf 		plot control} \\
\end{tabbing}

\begin{tabbing}
Titlexxxxxxxxxxxxxxx \= \kill
\underline{SYNTAX} \> {\bf 		``mode split"} \\
\> {\bf 	    or  ``mode nosplit"} \\
\end{tabbing}

\underline{DESCRIPTION}
\begin{list}{}{\setlength{\leftmargin}{0.5in}
     \setlength{\rightmargin}{0in}}
\item
Toggles the setting of the mode parameter, which determines
whether or not the data will be split into one plot per
source.  The default on startup is ``nosplit".
\end{list}
\vspace{.2in}

\begin{tabbing}
Titlexxxxxxxxxxxxxxx \= \kill
\underline{COMMAND} \> {\bf 	nfreq} \\
\end{tabbing}

\begin{tabbing}
Titlexxxxxxxxxxxxxxx \= \kill
\underline{TYPE} \> {\bf 		Data selection} \\
\end{tabbing}

\begin{tabbing}
Titlexxxxxxxxxxxxxxx \= \kill
\underline{SYNTAX} \> {\bf 	  e.g.	``nfreq $>$= 8"} \\
\> {\bf 	    or	``nfreq $<$ 2"} \\
\end{tabbing}

\underline{DESCRIPTION}
\begin{list}{}{\setlength{\leftmargin}{0.5in}
     \setlength{\rightmargin}{0in}}
\item
Sets the number of frequencies processed for this scan which
will pass the filtering functions applied in read, edit, and
plot.  The syntax is quite forgiving.  The requirements are
that there be an optional inequality operator, possibly followed 
by an equals sign, followed by a sensible number.  Spaces are 
irrelevant.  If the inequality is omitted, exactly the specified
number of frequencies must be present to pass the filters.
\item
If ``nfreq" is typed with no arguments, or with just ``0" as an 
argument, all restrictions on the number of frequencies 
processed are removed.
\item
\end{list}
\vspace{.2in}

\begin{tabbing}
Titlexxxxxxxxxxxxxxx \= \kill
\underline{COMMAND} \> {\bf 	nobatch} \\
\end{tabbing}

\begin{tabbing}
Titlexxxxxxxxxxxxxxx \= \kill
\underline{TYPE} \> {\bf 		Miscellaneous} \\
\end{tabbing}

\begin{tabbing}
Titlexxxxxxxxxxxxxxx \= \kill
\underline{SYNTAX} \> {\bf 		``nobatch"} \\
\end{tabbing}

\underline{DESCRIPTION}
\begin{list}{}{\setlength{\leftmargin}{0.5in}
     \setlength{\rightmargin}{0in}}
\item
Enables confirmation mechanism, for running interactively
(reverses the action of ``batch").
\end{list}
\vspace{.2in}

\begin{tabbing}
Titlexxxxxxxxxxxxxxx \= \kill
\underline{COMMAND} \> {\bf 	outversion} \\
\end{tabbing}

\begin{tabbing}
Titlexxxxxxxxxxxxxxx \= \kill
\underline{TYPE} \> {\bf 		IO control} \\
\end{tabbing}

\begin{tabbing}
Titlexxxxxxxxxxxxxxx \= \kill
\underline{SYNTAX} \> {\bf 		``outversion n", where n is an integer} \\
\end{tabbing}

\underline{DESCRIPTION}
\begin{list}{}{\setlength{\leftmargin}{0.5in}
     \setlength{\rightmargin}{0in}}
\item
This allows the user to override the output format of the A-file when
the write command is used.  Currently, only versions 1 and 2 are
supported.  If you specify version 0 (the default), each line will
be written individually with the same format as that in which it
originated.
\item
Note that writing data out in a different format version number from
the one it originated in will generate fields with undefined values
in the output.  Generally speaking, undefined strings are set to ``???",
and undefined numerical quantities are set to zero.
\end{list}
\vspace{.2in}

\begin{tabbing}
Titlexxxxxxxxxxxxxxx \= \kill
\underline{COMMAND} \> {\bf 	parameter} \\
\end{tabbing}

\begin{tabbing}
Titlexxxxxxxxxxxxxxx \= \kill
\underline{TYPE} \> {\bf 		Action} \\
\end{tabbing}

\begin{tabbing}
Titlexxxxxxxxxxxxxxx \= \kill
\underline{SYNTAX} \> {\bf 		``parameter 1 2 3 ..."	(non-interactive form)} \\
\> {\bf 		``parameter"		(interactive form)} \\
\end{tabbing}

\underline{DESCRIPTION}
\begin{list}{}{\setlength{\leftmargin}{0.5in}
     \setlength{\rightmargin}{0in}}
\item
This command causes all unedited type 2 data in memory to be treated as the
basis for a parameter extraction operation from disk-resident type-2 files.
Specified parameters are placed in a special array attached to each type-2 line
in memory.  The parameter specification is via key numbers.  These numbers
may be specified either directly on the input line of the parameter command,
or in response to a query from the program if no parameter keys are given.
In batch mode, aedit assumes that the former mechanism is being used, and
the absence of any keys is treated as an error.  Once extracted, the parameters
may be written to a file of the user`s choice, using the pwrite command.
\item
Obviously, aedit cannot extract parameters unless the relevant type-2 files are
on the disk.  Make sure the DATADIR environment variable is pointing to the 
correct data area.
\item
Each invocation of the parameter command obliterates all previous parameters
extracted for a previous subset of unflagged data lines.
\item
Below is a list of the available parameters, and their index numbers which
must be supplied in a space-delimited list.  The total number of parameters 
allowed is currently 32, and each array of parameters (denoted by the parentheses 
below) counts one for each array element An index number in parentheses indicates 
that the parameter is already in memory, but can be selected as a parameter 
for manipulation and output like the others
\item
\begin{verbatim}
INDEX  PARAMETER NAME                   INDEX  PARAMETER NAME
-----  --------------                   -----  --------------
 1:    ref_pcal_amp (6)                 29:    yperror
 2:    ref_pcal_phase (6)               30:    suppress
 3:    ref_pcal_diff (6)                31:    ppupdate
 4:    ref_pcal_freq (6)                32:    xslip
 5:    ref_pcal_rate                    33:    yslip
 6:    rem_pcal_amp (6)                 34:    badsync
 7:    rem_pcal_phase (6)               35:    ref_drive
 8:    rem_pcal_diff (6)                36:    rem_drive
 9:    rem_pcal_freq (6)                (51):  scan_length
10:    rem_pcal_rate                    (52):  scantime
11:    errate_ref_usb (6)               (53):  amplitude
12:    errate_ref_lsb (6)               (54):  snr
13:    errate_rem_usb (6)               (55):  phase
14:    errate_rem_lsb (6)               (56):  resid_sbd
15:    corel_amp (6)                    (57):  resid_mbd
16:    corel_phase (6)                  (58):  ambiguity
17:    rate_error                       (59):  resid_rate
18:    mbdelay_error                    (60):  ref_elevation
19:    sbdelay_error                    (61):  rem_elevation
20:    total_phase                      (62):  ref_azimuth
21:    tot_phase_mid                    (63):  rem_azimuth
22:    incoherent_amp                   (64):  u
23:    mhz_arcsec_ns                    (65):  v
24:    mhz_arcsec_ew                    (66):  ref_frequency
25:    pcnt_discard                     (67):  total_ec_phase
26:    min_max_ratio                    (68):  total_rate
27:    lo_frequency (6)                 (69):  total_mbd
28:    xperror                          (70):  total_sbd-mbd
\end{verbatim}
\end{list}
\vspace{.2in}

\begin{tabbing}
Titlexxxxxxxxxxxxxxx \= \kill
\underline{COMMAND} \> {\bf 	plist} \\
\end{tabbing}

\begin{tabbing}
Titlexxxxxxxxxxxxxxx \= \kill
\underline{TYPE} \> {\bf 		action} \\
\end{tabbing}

\begin{tabbing}
Titlexxxxxxxxxxxxxxx \= \kill
\underline{SYNTAX} \> {\bf 		``plist"} \\
\end{tabbing}

\underline{DESCRIPTION}
\begin{list}{}{\setlength{\leftmargin}{0.5in}
     \setlength{\rightmargin}{0in}}
\item
This command summarizes the state of extracted parameters in memory.  Various
states can exist, with varying degrees of overlap between flagged and unflagged
records, with and without attached extracted parameters.  This command is
provided to remind the user of the degree to which he/she has confused
him/herself.
\item
A more important function is to attach a numerical identifying tag to each
extracted parameter present.  This tag is then used to identify the parameter
to be examined in filtering operations, using the ``prange" command, or the
parameter to plot, in eth ``axis" command.  You can see whether you got the 
specification correct by using the ``inputs" command after attempting to use 
``prange", or by looking at the axis labels on the plot.
\item
The id tag is the first field in the output of ``plist".
\end{list}
\vspace{.2in}

\begin{tabbing}
Titlexxxxxxxxxxxxxxx \= \kill
\underline{COMMAND} \> {\bf 	plot} \\
\end{tabbing}

\begin{tabbing}
Titlexxxxxxxxxxxxxxx \= \kill
\underline{TYPE} \> {\bf 		Action} \\
\end{tabbing}

\begin{tabbing}
Titlexxxxxxxxxxxxxxx \= \kill
\underline{SYNTAX} \> {\bf 		``plot $<$y-axis$>$ $<$x-axis$>$"} \\
\end{tabbing}

\underline{DESCRIPTION}
\begin{list}{}{\setlength{\leftmargin}{0.5in}
     \setlength{\rightmargin}{0in}}
\item
Initiates plotting of data in memory on a device of the
users choice.  The data are divided into reasonable logical
units (such as stations, baselines, triangles etc.) before
plotting.  Only one experiment/frequency combination is
plotted on any given page, though such a combination may
span many pages.  If the input parameter ``mode" is set to
``split", as opposed to ``nosplit", a separate set of plots 
is generated for each source present.  
\item
The data are filtered by the input settings
before plotting takes place, so you can plot restrictively
without having to actually edit the data.
\item
The optional arguments ``y-axis" and ``x-axis" determine what
variables get plotted against each other.  If these
arguments are omitted, the axis input settings (which
can be set either in the previous ``plot" command, or in
a separate ``axis" command) will be used.  If only one axis
is specified, it is assumed to be the Y axis, and the X axis
is set to ``scantime".  Certain combinations of axes are
nonsensical, and are locked out.  For a list of available
plot axes, see ``help axis".
\item
The behaviour of the plot command is controlled by a few other
parameters.  The ``grid" input setting determines how many
plots will appear per page in the x and y directions.  The
``xscale" and ``yscale" input parameters allow user-override
of the default range of the plots (normally either the natural
range of the data, or a fixed range for phase-like quantities).
Scan time is handled slightly differently, in that all plots
in a frequency/experiment combination are forced to the same
start and stop times on the plot, to ensure that plots line
up with each other.  Manual override of the default is
accomplished via the ``timerange" input setting.
\item
The ``device" input setting determines what plot device will be
used for the plots.  If you leave this blank, aedit will prompt
you with a list of accessible devices.  For details, see
``help device".  Note also that you must issue a ``clear plot"
command to make sure aedit has finished writing (buffered)
information to the plot.  This is crucial for hardcopy devices,
as the disk file generated by aedit (which must then be manually
sent to an appropriate printer) will be incomplete otherwise.
\item
If you are using an interactive device, particularly on an
X-window workstation, you will be able to perform point-and-shoot
editing, area editing (see ``help edit"), and identification
and examination of individual data points (see ``help fplot" and
``help zoom").  To use these features, you must of course have
data plotted on the screen.
\end{list}
\vspace{.2in}

\begin{tabbing}
Titlexxxxxxxxxxxxxxx \= \kill
\underline{COMMAND} \> {\bf 	prange} \\
\end{tabbing}

\begin{tabbing}
Titlexxxxxxxxxxxxxxx \= \kill
\underline{TYPE} \> {\bf 		Data selection} \\
\end{tabbing}

\begin{tabbing}
Titlexxxxxxxxxxxxxxx \= \kill
\underline{SYNTAX} \> {\bf 		e.g. ``prange 2 $>$4.6"} \\
\> {\bf 		or ``prange 17 $<$1.5e-12"} \\
\> {\bf 		or ``prange 1 -25 74"} \\
\end{tabbing}

\underline{DESCRIPTION}
\begin{list}{}{\setlength{\leftmargin}{0.5in}
     \setlength{\rightmargin}{0in}}
\item
This command sets the input filter for a selected extracted parameter.  The
parameter is identified by the first argument, which is the identification tag
of the parameter reported by the ``plist" command.  It is thus not easy to run
``prange" without first executing ``plist".  Neither of these commands work,
obviously, unless you have already extracted some parameters with the
``parameter" command.
\item
The parameter data range of this filter can be specified either with an inequality
(no $>$= or $<$= because all parameters are floating point quantities internally),
or two numerical values, a lower and then an upper limit.  To exclude a finite
range of values, you must merge two input files, each of which has had one of
the inequality limits applied.  This limitation will be removed in due course.
Thus, the above examples will pass, respectively:
\item
All scans with values of extracted parameter 2 greater than 4.6
\item
All scans with values of extracted parameter 17 less than 1.5e-12
\item
All scans with values of extracted parameter 1 between -25.0 and +74.0
\item
In the same manner as all filter settings in aedit, the data flagging occurs only
upon invocation of the ``edit inputs" command.  In the case of extracted
parameters, the filtering during a read operation which normally occurs is
suppressed.
\end{list}
\vspace{.2in}

\begin{tabbing}
Titlexxxxxxxxxxxxxxx \= \kill
\underline{COMMAND} \> {\bf 	procrange} \\
\end{tabbing}

\begin{tabbing}
Titlexxxxxxxxxxxxxxx \= \kill
\underline{TYPE} \> {\bf 		Data selection} \\
\end{tabbing}

\begin{tabbing}
Titlexxxxxxxxxxxxxxx \= \kill
\underline{SYNTAX} \> {\bf 		``procrange yyddd-hhmm yyddd-hhmm"} \\
\end{tabbing}

\underline{DESCRIPTION}
\begin{list}{}{\setlength{\leftmargin}{0.5in}
     \setlength{\rightmargin}{0in}}
\item
Sets the range of procdates outside which data will be rejected by
various filter-applying action commands (edit inputs, read).
Typing ``procrange" without arguments removes any restriction on the
procdate range.
\end{list}
\vspace{.2in}

\begin{tabbing}
Titlexxxxxxxxxxxxxxx \= \kill
\underline{COMMAND} \> {\bf 	psfile} \\
\end{tabbing}

\begin{tabbing}
Titlexxxxxxxxxxxxxxx \= \kill
\underline{TYPE} \> {\bf 		Experiment summary} \\
\end{tabbing}

\begin{tabbing}
Titlexxxxxxxxxxxxxxx \= \kill
\underline{SYNTAX} \> {\bf 		``psfile filename"} \\
\end{tabbing}

\underline{DESCRIPTION}
\begin{list}{}{\setlength{\leftmargin}{0.5in}
     \setlength{\rightmargin}{0in}}
\item
This command is not yet implemented.  Sorry.
\end{list}
\vspace{.2in}

\begin{tabbing}
Titlexxxxxxxxxxxxxxx \= \kill
\underline{COMMAND} \> {\bf 	psplot} \\
\end{tabbing}

\begin{tabbing}
Titlexxxxxxxxxxxxxxx \= \kill
\underline{TYPE} \> {\bf 		Experiment summary} \\
\end{tabbing}

\begin{tabbing}
Titlexxxxxxxxxxxxxxx \= \kill
\underline{SYNTAX} \> {\bf 		``psplot"} \\
\end{tabbing}

\underline{DESCRIPTION}
\begin{list}{}{\setlength{\leftmargin}{0.5in}
     \setlength{\rightmargin}{0in}}
\item
This command takes the data in memory plus the image of a cc file in memory
and constructs a 2-D array of quality codes.  This is then displayed
in an Xwindow PGPLOT window as a (possibly multi-page) colour-coded
matrix.  Interactive cursor operations are then invoked for data
perusal, editing, and fringe plot popups.
\item
The on-screen buttons are self-explanatory.  In general, the left mouse
button either tags or pops up fringe plots (depending on the setting
of the on-screen buttons), while the middle button prints cell identification
information in the lower left corner of the window.  The right button
immediately zaps the data point from the database.  Tagged cells are
indicated by a small white triangle in the center of the cell.  The
keystrokes `a', `f' and `x' cause tagging, fringe plot popup, and
immediate zapping respectively, regardless of the setting of the on-screen
buttons.  You can tag many cells at once.  To tag an entire baseline,
click on the baseline label.  To tag an entire scan, click on the scan
label.  To tag all cells with a particular quality code, click on the
quality code key at the bottom of the window.  All tagging is done on
a toggle basis (i.e. do it again and the tag disappears).  Upon exit of
psplot, you can zap, write out to disk, or ignore the list of tagged
records.
\end{list}
\vspace{.2in}

\begin{tabbing}
Titlexxxxxxxxxxxxxxx \= \kill
\underline{COMMAND} \> {\bf 	pwrite} \\
\end{tabbing}

\begin{tabbing}
Titlexxxxxxxxxxxxxxx \= \kill
\underline{TYPE} \> {\bf 		Action} \\
\end{tabbing}

\begin{tabbing}
Titlexxxxxxxxxxxxxxx \= \kill
\underline{SYNTAX} \> {\bf 		``pwrite filename"} \\
\end{tabbing}

\underline{DESCRIPTION}
\begin{list}{}{\setlength{\leftmargin}{0.5in}
     \setlength{\rightmargin}{0in}}
\item
Writes all unflagged user-extracted parameter data in memory out 
to the filename specified in the argument.  The data are written out
according to the current sort order (as determined by
execution of the ``sort" command).  If the data are not
sorted, the output order is the same as the order in
which the data were read.  You must execute the parameter command
before using pwrite.  Unflagged data lines which for any reason
do not have associated extracted parameters are ignored by pwrite.
\item
The list of user-extracted parameters is preceded by information
identifying the baseline, scan and extent number, together with
a few other generally useful items (but far less than is present
in the A-file format).
\end{list}
\vspace{.2in}

\begin{tabbing}
Titlexxxxxxxxxxxxxxx \= \kill
\underline{COMMAND} \> {\bf 	qcodes} \\
\end{tabbing}

\begin{tabbing}
Titlexxxxxxxxxxxxxxx \= \kill
\underline{TYPE} \> {\bf 		Data selection} \\
\end{tabbing}

\begin{tabbing}
Titlexxxxxxxxxxxxxxx \= \kill
\underline{SYNTAX} \> {\bf 		``qcodes 5,6,789,DEF"} \\
\> {\bf 	    or  ``qcodes 5-9 D-F"} \\
\> {\bf 	    or  ``qcodes not 0-4 A-C"} \\
\end{tabbing}

\underline{DESCRIPTION}
\begin{list}{}{\setlength{\leftmargin}{0.5in}
     \setlength{\rightmargin}{0in}}
\item
Sets the quality code data selection input parameter.  Shown
in the example are three ways of establishing the quality code
filter ``56789DEF".  You can specify codes directly, in any order,
separated by spaces, commas, or nothing at all.  You can also
specify ranges of quality codes from the sequence ``ABCDEF0123456789"
by using the construction ``2-8".  Preceding a specification by the
exact string ``not" means take all except the specified codes.  This
information is applied as a filter by edit inputs, read, and plot.
Typing ``qcodes" with no argument removes any limitation on quality
codes.
\end{list}
\vspace{.2in}

\begin{tabbing}
Titlexxxxxxxxxxxxxxx \= \kill
\underline{COMMAND} \> {\bf 	read} \\
\end{tabbing}

\begin{tabbing}
Titlexxxxxxxxxxxxxxx \= \kill
\underline{TYPE} \> {\bf 		Action} \\
\end{tabbing}

\begin{tabbing}
Titlexxxxxxxxxxxxxxx \= \kill
\underline{SYNTAX} \> {\bf 		``read filename"} \\
\end{tabbing}

\underline{DESCRIPTION}
\begin{list}{}{\setlength{\leftmargin}{0.5in}
     \setlength{\rightmargin}{0in}}
\item
Reads data in from the filename specified in the argument.
If enough fields on the line are successfully decoded to 
identify the parent data file, a data entry is made in memory.  
If not even enough could be decoded to id the file, the line 
is skipped.  You can read as many files into aedit, one after 
the other, as you like.  ``Read" filters the incoming data 
according to the data selection input parameters.
\item
The unlimited data capacity of aedit is achieved by using
dynamic memory allocation inside the ``read" function.  As
more memory is needed, the program obtains it from the system.
This memory is released by the command ``clear data", or
by ``exit".  The user is informed of memory usage during the
reading operation.
\end{list}
\vspace{.2in}

\begin{tabbing}
Titlexxxxxxxxxxxxxxx \= \kill
\underline{COMMAND} \> {\bf 	reference} \\
\end{tabbing}

\begin{tabbing}
Titlexxxxxxxxxxxxxxx \= \kill
\underline{TYPE} \> {\bf 		Plot control} \\
\end{tabbing}

\begin{tabbing}
Titlexxxxxxxxxxxxxxx \= \kill
\underline{SYNTAX} \> {\bf 		``reference"} \\
\end{tabbing}

\underline{DESCRIPTION}
\begin{list}{}{\setlength{\leftmargin}{0.5in}
     \setlength{\rightmargin}{0in}}
\item
When a station-based quantity, like elevation, is plotted against a
baseline-based quantity, like SNR, either the reference or remote
elevation must be used.  This command specifies that it should be
the reference quantity which gets plotted, which is the startup
default.  There is a corresponding ``remote" command.
\end{list}
\vspace{.2in}

\begin{tabbing}
Titlexxxxxxxxxxxxxxx \= \kill
\underline{COMMAND} \> {\bf 	remote} \\
\end{tabbing}

\begin{tabbing}
Titlexxxxxxxxxxxxxxx \= \kill
\underline{TYPE} \> {\bf 		Plot control} \\
\end{tabbing}

\begin{tabbing}
Titlexxxxxxxxxxxxxxx \= \kill
\underline{SYNTAX} \> {\bf 		``remote"} \\
\end{tabbing}

\underline{DESCRIPTION}
\begin{list}{}{\setlength{\leftmargin}{0.5in}
     \setlength{\rightmargin}{0in}}
\item
When a station-based quantity, like elevation, is plotted against a
baseline-based quantity, like SNR, iether the reference or remote
elevation must be used.  This command specifies that it should be
the remote quantity which gets plotted.  There is a corresponding
``reference" command, the results of which are the startup default.
\end{list}
\vspace{.2in}

\begin{tabbing}
Titlexxxxxxxxxxxxxxx \= \kill
\underline{COMMAND} \> {\bf 	run} \\
\end{tabbing}

\begin{tabbing}
Titlexxxxxxxxxxxxxxx \= \kill
\underline{TYPE} \> {\bf 		action} \\
\end{tabbing}

\begin{tabbing}
Titlexxxxxxxxxxxxxxx \= \kill
\underline{SYNTAX} \> {\bf 		``run filename"} \\
\end{tabbing}

\underline{DESCRIPTION}
\begin{list}{}{\setlength{\leftmargin}{0.5in}
     \setlength{\rightmargin}{0in}}
\item
Causes the aedit commands in ``filename" to be executed, just
as if they were typed at the terminal.  For obvious reasons,
there are a couple of exceptions.  Confirmation is no longer
requested when using run files, and certain interactive
operations (edit cursor, zoom) are disabled.  Aedit command
files can be nested up to a depth of 10.  Any error within
a run file causes the execution to abort, and control returns
to the terminal, regardless of how deeply the runfiles are
nested.
\end{list}
\vspace{.2in}

\begin{tabbing}
Titlexxxxxxxxxxxxxxx \= \kill
\underline{COMMAND} \> {\bf 	setyear} \\
\end{tabbing}

\begin{tabbing}
Titlexxxxxxxxxxxxxxx \= \kill
\underline{TYPE} \> {\bf 		Action} \\
\end{tabbing}

\begin{tabbing}
Titlexxxxxxxxxxxxxxx \= \kill
\underline{SYNTAX} \> {\bf 		``setyear 1989"} \\
\end{tabbing}

\underline{DESCRIPTION}
\begin{list}{}{\setlength{\leftmargin}{0.5in}
     \setlength{\rightmargin}{0in}}
\item
This command is present only to allow the user to circumvent an
unfortunate problem with the A-file format.  Some A-files have
the year of the scan in field 7, but in others this information
is replaced by the number of the parent type-51 HP-1000 extent.
Generally, aedit will recognize the latter type of A-file on
read, and notify the user that the scan year information is
missing from some of the data.  In such cases, the year is set
to 1980.  The recommended course of action is for the user to
set the timerange to the offending span in 1980 with all other
filters wide open, run ``edit inputs" to flag all good data,
force the year to the correct value with ``setyear", and unflag
the good data again.  If all data is actually from the same
calendar year, the edit and unflag steps are unnecessary - you
can run setyear on the whole dataset.
\item
Confusing things could happen if the parent extent number exceeds
80, but this should be almost never.
\item
This command should become obsolete as the move to UNIX proceeds.
\end{list}
\vspace{.2in}

\begin{tabbing}
Titlexxxxxxxxxxxxxxx \= \kill
\underline{COMMAND} \> {\bf 	snrmax} \\
\end{tabbing}

\begin{tabbing}
Titlexxxxxxxxxxxxxxx \= \kill
\underline{TYPE} \> {\bf 		Data selection} \\
\end{tabbing}

\begin{tabbing}
Titlexxxxxxxxxxxxxxx \= \kill
\underline{SYNTAX} \> {\bf 		``snrmax 20"} \\
\end{tabbing}

\underline{DESCRIPTION}
\begin{list}{}{\setlength{\leftmargin}{0.5in}
     \setlength{\rightmargin}{0in}}
\item
Sets the maximum snr which will pass the filters applied
in edit inputs, read, and plot.  Typing ``snrmax" without
arguments removes any upper bound on snr.
\end{list}
\vspace{.2in}

\begin{tabbing}
Titlexxxxxxxxxxxxxxx \= \kill
\underline{COMMAND} \> {\bf 	snrmin} \\
\end{tabbing}

\begin{tabbing}
Titlexxxxxxxxxxxxxxx \= \kill
\underline{TYPE} \> {\bf 		Data selection} \\
\end{tabbing}

\begin{tabbing}
Titlexxxxxxxxxxxxxxx \= \kill
\underline{SYNTAX} \> {\bf 		``snrmin 20"} \\
\end{tabbing}

\underline{DESCRIPTION}
\begin{list}{}{\setlength{\leftmargin}{0.5in}
     \setlength{\rightmargin}{0in}}
\item
Sets the minimum snr which will pass the filters applied
in edit inputs, read, and plot.  Typing ``snrmin" without
arguments removes any lower bound on snr.
\end{list}
\vspace{.2in}

\begin{tabbing}
Titlexxxxxxxxxxxxxxx \= \kill
\underline{COMMAND} \> {\bf 	sort} \\
\end{tabbing}

\begin{tabbing}
Titlexxxxxxxxxxxxxxx \= \kill
\underline{TYPE} \> {\bf 		Action} \\
\end{tabbing}

\begin{tabbing}
Titlexxxxxxxxxxxxxxx \= \kill
\underline{SYNTAX} \> {\bf 		``sort scantime"} \\
\> {\bf 		``sort procdate"} \\
\> {\bf 		``sort snr"} \\
\> {\bf 		``sort length"} \\
\> {\bf 		``sort baseline"} \\
\> {\bf 		``sort triangle"} \\
\> {\bf 		``sort frequency"} \\
\> {\bf 		``sort sourcename"} \\
\> {\bf 		``sort qcode"} \\
\> {\bf 		``sort experiment"} \\
\> {\bf 		``sort rootcode"} \\
\> {\bf 		``sort family"} \\
\end{tabbing}

\underline{DESCRIPTION}
\begin{list}{}{\setlength{\leftmargin}{0.5in}
     \setlength{\rightmargin}{0in}}
\item
Sorts the data in memory according the the value of the
field specified in the command.  The sort is stable, in that
entries in the database which compare equal will retain
their original sort order.  Thus any combination of sort
key priorities can be applied by repeated execution of ``sort".
\item
The only consequence of sorting the data is that the output
of the ``write" command will be sorted.  The various ``aedit"
commands do not care whether the data are sorted or not, and in
fact operate on unsorted data whether sort has been executed or
not.  Flagging and unflagging data does not affect the sort
order ... sorting is done on all the data.
\item
The sort keys ``snr", ``length" and ``frequency" have no effect on
root or corel data.  In addition, the sort keys ``baseline" and
``qcode" have no effect on root data.  The ``snr" sort key uses
the bispectral snr for triangle records.
\item
Note that reading in additional data destroys any sort order.  The
effect is the same as issuing the ``unsort" command, in that
the sort order information for the original dataset is explicitly
discarded, and the original sort order restored.
\item
The special key ``family" causes the ``write" command to write data
to disk grouped by root family.  Each family has the root record
(if present), followed by the corel records (if present), followed
by the fringe records (if present).
\end{list}
\vspace{.2in}

\begin{tabbing}
Titlexxxxxxxxxxxxxxx \= \kill
\underline{COMMAND} \> {\bf 	sources} \\
\end{tabbing}

\begin{tabbing}
Titlexxxxxxxxxxxxxxx \= \kill
\underline{TYPE} \> {\bf 		Data selection} \\
\end{tabbing}

\begin{tabbing}
Titlexxxxxxxxxxxxxxx \= \kill
\underline{SYNTAX} \> {\bf 		``sources 3C345, 3C273, OJ287"} \\
\end{tabbing}

\underline{DESCRIPTION}
\begin{list}{}{\setlength{\leftmargin}{0.5in}
     \setlength{\rightmargin}{0in}}
\item
Specifies a list of sources which will pass the filters in
edit inputs, read, and plot.  The source names must match
those in the data files exactly (including case), with the
exception of leading or trailing blanks.  Typing ``sources"
with no arguments removes any restriction on sources.
\end{list}
\vspace{.2in}

\begin{tabbing}
Titlexxxxxxxxxxxxxxx \= \kill
\underline{COMMAND} \> {\bf 	stations} \\
\end{tabbing}

\begin{tabbing}
Titlexxxxxxxxxxxxxxx \= \kill
\underline{TYPE} \> {\bf 		Data selection} \\
\end{tabbing}

\begin{tabbing}
Titlexxxxxxxxxxxxxxx \= \kill
\underline{SYNTAX} \> {\bf 		``stations ABC D EF"} \\
\end{tabbing}

\underline{DESCRIPTION}
\begin{list}{}{\setlength{\leftmargin}{0.5in}
     \setlength{\rightmargin}{0in}}
\item
Enters a list of allowed stations into the inputs.  All alphabetic
characters are accepted, in any order, lower or upper case, with or
without spaces or commas.  Duplicate characters are ignored.  Baselines
which involve stations not in this list will fail the filter tests
applied by edit inputs, read, and plot.  Typing ``stations" without
arguments removes any limitations on stations.
\end{list}
\vspace{.2in}

\begin{tabbing}
Titlexxxxxxxxxxxxxxx \= \kill
\underline{COMMAND} \> {\bf    summary} \\
\end{tabbing}

\begin{tabbing}
Titlexxxxxxxxxxxxxxx \= \kill
\underline{TYPE} \> {\bf            Action} \\
\end{tabbing}

\begin{tabbing}
Titlexxxxxxxxxxxxxxx \= \kill
\underline{SYNTAX} \> {\bf          ``summary"} \\
\> {\bf                 ``summary 0" or ``summary root"} \\
\> {\bf                 ``summary 1" or ``summary corel"} \\
\> {\bf                 ``summary 2" or ``summary fringe"} \\
\> {\bf                 ``summary 3" or ``summary triangle"} \\
\end{tabbing}

\underline{DESCRIPTION}
\begin{list}{}{\setlength{\leftmargin}{0.5in}
     \setlength{\rightmargin}{0in}}
\item
Displays a summary of all unflagged data in memory on
the terminal.  Without arguments, a terse summary of all data of
all types is given.  If a record type is specified, much more
detailed information is provided.  An example is given below.
\item
\begin{verbatim}
                SUMMARY OF UNFLAGGED DATA IN MEMORY
                -----------------------------------

Total number of unflagged fringe records = 6754

Earliest scan:       94-015-183000
Latest scan:         94-016-181505
Earliest procdate:   94-050-1648
Latest procdate:     94-055-1044
Stations present:    DAKLETV
Baselines present:   DA DK DL DE AK AL AE KL KE LE TE AT AV TV EV KT KV DV LV DT LT
Frequencies present: XS
SNR extrema:         0.000  1069.
Experiments present: 2498
Sources present:     0048-097 0059+581 0119+041 0229+131 0454-234 0458-020
        0528+134 0537-441 0552+398 0727-115 0735+178 0804+499 0820+560
        0823+033 0919-260 0954+658 0955+476 1034-293 1044+719 1053+815
        1104-445 1128+385 1219+044 1308+326 1334-127 1357+769 1424-418
        1606+106 1622-253 1726+455 1739+522 1741-038 1749+096 1921-293
        2145+067 2234+282 2255-282 4C39.25 NRAO512 OJ287 OK290
Quality code summary:
        A B C D   E F  0  1 2  3 4  5  6   7   8   9    ?
        0 2 0 137 3 93 88 0 46 5 18 60 144 211 851 5096 0

There are 0 flagged records present
\end{verbatim}
\end{list}
\vspace{.2in}

\begin{tabbing}
Titlexxxxxxxxxxxxxxx \= \kill
\underline{COMMAND} \> {\bf 	timerange} \\
\end{tabbing}

\begin{tabbing}
Titlexxxxxxxxxxxxxxx \= \kill
\underline{TYPE} \> {\bf 		Data selection} \\
\end{tabbing}

\begin{tabbing}
Titlexxxxxxxxxxxxxxx \= \kill
\underline{SYNTAX} \> {\bf 		``timerange yyddd-hhmmss yyddd-hhmmss"} \\
\end{tabbing}

\underline{DESCRIPTION}
\begin{list}{}{\setlength{\leftmargin}{0.5in}
     \setlength{\rightmargin}{0in}}
\item
Sets the range of times outside which data will be rejected by
various filter-applying action commands (edit inputs, read).
Overrides self-scaling of the time axis on plots.  Typing
``timerange" without arguments removes any restriction on the
timerange.
\end{list}
\vspace{.2in}

\begin{tabbing}
Titlexxxxxxxxxxxxxxx \= \kill
\underline{COMMAND} \> {\bf 	type} \\
\end{tabbing}

\begin{tabbing}
Titlexxxxxxxxxxxxxxx \= \kill
\underline{TYPE} \> {\bf 		Data selection} \\
\end{tabbing}

\begin{tabbing}
Titlexxxxxxxxxxxxxxx \= \kill
\underline{SYNTAX} \> {\bf 		``type 0 1 2 3"} \\
\end{tabbing}

\underline{DESCRIPTION}
\begin{list}{}{\setlength{\leftmargin}{0.5in}
     \setlength{\rightmargin}{0in}}
\item
Sets the type data selection parameter.  Only records of the
specified type(s) will pass the data editing functions in ``read",
``edit" and ``plot".  The digits 0, 1, 2 and 3 can occur anywhere in the 
argument.  Omission of the arguments implies ``type 0123".
\end{list}
\vspace{.2in}

\begin{tabbing}
Titlexxxxxxxxxxxxxxx \= \kill
\underline{COMMAND} \> {\bf 	unflag} \\
\end{tabbing}

\begin{tabbing}
Titlexxxxxxxxxxxxxxx \= \kill
\underline{TYPE} \> {\bf 		Action} \\
\end{tabbing}

\begin{tabbing}
Titlexxxxxxxxxxxxxxx \= \kill
\underline{SYNTAX} \> {\bf 		``unflag all"} \\
\> {\bf 		``unflag duplicates"} \\
\> {\bf 		``unflag cursor"} \\
\> {\bf 		``unflag qcodes"} \\
\> {\bf 		``unflag snr"} \\
\> {\bf 		``unflag bsnr"} \\
\> {\bf 		``unflag timerange"} \\
\> {\bf 		``unflag procrange"} \\
\> {\bf 		``unflag stations"} \\
\> {\bf 		``unflag baselines"} \\
\> {\bf 		``unflag triangles"} \\
\> {\bf 		``unflag experiment"} \\
\> {\bf 		``unflag frequencies"} \\
\> {\bf 		``unflag type"} \\
\> {\bf 		``unflag sources"} \\
\> {\bf 		``unflag length"} \\
\> {\bf 		``unflag fraction"} \\
\> {\bf 		``unflag nfreq"} \\
\> {\bf 		``unflag parameter"} \\
\> {\bf 		``unflag parents"} \\
\> {\bf 		``unflag children"} \\
\> {\bf 		``unflag nobaselines"} \\
\> {\bf 		``unflag notriangles"} \\
\end{tabbing}

\underline{DESCRIPTION}
\begin{list}{}{\setlength{\leftmargin}{0.5in}
     \setlength{\rightmargin}{0in}}
\item
Unsets flag bits for all scans in memory, according to the
argument.  For example, ``unflag snrmin" unsets all flag bits
thoughout memory which were set with an ``edit inp" with
``snrmin" set higher than the scan snr value.  The combination
of ``edit" and ``unflag" allows great control over the flagging
status of the data in memory.  ``Unflag all" removes all flags
from the data simultaneously.
\end{list}
\vspace{.2in}

\begin{tabbing}
Titlexxxxxxxxxxxxxxx \= \kill
\underline{COMMAND} \> {\bf 	unsort} \\
\end{tabbing}

\begin{tabbing}
Titlexxxxxxxxxxxxxxx \= \kill
\underline{TYPE} \> {\bf 		Action} \\
\end{tabbing}

\begin{tabbing}
Titlexxxxxxxxxxxxxxx \= \kill
\underline{SYNTAX} \> {\bf 		``unsort"} \\
\end{tabbing}

\underline{DESCRIPTION}
\begin{list}{}{\setlength{\leftmargin}{0.5in}
     \setlength{\rightmargin}{0in}}
\item
Restores the original sort order of the data (i.e. the
order in which the data was read in).
\end{list}
\vspace{.2in}

\begin{tabbing}
Titlexxxxxxxxxxxxxxx \= \kill
\underline{COMMAND} \> {\bf 	write} \\
\end{tabbing}

\begin{tabbing}
Titlexxxxxxxxxxxxxxx \= \kill
\underline{TYPE} \> {\bf 		Action} \\
\end{tabbing}

\begin{tabbing}
Titlexxxxxxxxxxxxxxx \= \kill
\underline{SYNTAX} \> {\bf 		``write filename"} \\
\end{tabbing}

\underline{DESCRIPTION}
\begin{list}{}{\setlength{\leftmargin}{0.5in}
     \setlength{\rightmargin}{0in}}
\item
Writes all unflagged data in memory out to the filename
specified in the argument.  The data are written out
according to the current sort order (as determined by
execution of the ``sort" command).  If the data are not
sorted, the output order is the same as the order in
which the data were read.  The output A-file format version
is controlled by the ``outversion" command, and defualts to 
the same as that read in.
\item
The input filters are ignored on ``write".
The way to write out selected data is to set the input
filters and then run ``edit inputs", before using ``write".
\end{list}
\vspace{.2in}

\begin{tabbing}
Titlexxxxxxxxxxxxxxx \= \kill
\underline{COMMAND} \> {\bf 	xscale} \\
\end{tabbing}

\begin{tabbing}
Titlexxxxxxxxxxxxxxx \= \kill
\underline{TYPE} \> {\bf 		Plot control} \\
\end{tabbing}

\begin{tabbing}
Titlexxxxxxxxxxxxxxx \= \kill
\underline{SYNTAX} \> {\bf 		``xscale xmin xmax" (xmin, xmax floating point)} \\
\end{tabbing}

\underline{DESCRIPTION}
\begin{list}{}{\setlength{\leftmargin}{0.5in}
     \setlength{\rightmargin}{0in}}
\item
Sets the minimum and maximum X-axis values, overriding the
default, which is 0 to 360 degrees for phase quantities, or
the data range plus 10\% at each end for other quantities.
Scale with no arguments restores the default.  When
the X-axis is multiband delay, the scale is automatically set
to +/- half the multiband delay ambiguity.
This can be overridden by an explicit ``xscale" command.
If the axis is scantime, the ``timerange" settings determine
the axis extrema.
\item
Points which fall outside the scale limits are not plotted, and
a warning message is issued to alert the user as to how many
points were omitted.
\end{list}
\vspace{.2in}

\begin{tabbing}
Titlexxxxxxxxxxxxxxx \= \kill
\underline{COMMAND} \> {\bf 	yscale} \\
\end{tabbing}

\begin{tabbing}
Titlexxxxxxxxxxxxxxx \= \kill
\underline{TYPE} \> {\bf 		Plot control} \\
\end{tabbing}

\begin{tabbing}
Titlexxxxxxxxxxxxxxx \= \kill
\underline{SYNTAX} \> {\bf 		``yscale ymin ymax" (ymin, ymax floating point)} \\
\end{tabbing}

\underline{DESCRIPTION}
\begin{list}{}{\setlength{\leftmargin}{0.5in}
     \setlength{\rightmargin}{0in}}
\item
Sets the minimum and maximum Y-axis values, overriding the
default, which is 0 to 360 degrees for phase quantities, or
the data range plus 10\% at each end for other quantities.
Scale with no arguments restores the default.  When
the Y-axis is multiband delay, the scale is automatically set
to +/- half the multiband delay ambiguity
This can be overridden by an explicit ``yscale" command.
If the axis is scantime, the ``timerange" settings determine
the axis extrema.
\item
Points which fall outside the scale limits are not plotted, and
a warning message is issued to alert the user as to how many
points were omitted.
\end{list}
\vspace{.2in}

\begin{tabbing}
Titlexxxxxxxxxxxxxxx \= \kill
\underline{COMMAND} \> {\bf 	zoom} \\
\end{tabbing}

\begin{tabbing}
Titlexxxxxxxxxxxxxxx \= \kill
\underline{TYPE} \> {\bf 		Action} \\
\end{tabbing}

\begin{tabbing}
Titlexxxxxxxxxxxxxxx \= \kill
\underline{SYNTAX} \> {\bf 		``zoom" (no arguments)} \\
\end{tabbing}

\underline{DESCRIPTION}
\begin{list}{}{\setlength{\leftmargin}{0.5in}
     \setlength{\rightmargin}{0in}}
\item
Enables the cursor on an active plot on an interactive
graphics device.  The user selects a point by positioning
the cursor and typing any character except `x' or `X', and 
the program displays detailed information about that point 
on the terminal.  The only information displayed 
is that resident in memory (i.e. A-file information). If you
want more detail, make sure the binary data files are on
disk, set the DATADIR variable appropriately, and use the
``fplot" command to pop up a fringe plot on your Xwindow
screen.
\item
``Zoom" is terminated by typing an `x' or `X'.
\end{list}
\vspace{.2in}

\end{document}
